\section{Multi-jet QCD estimation} \label{sec:background:qcd}

The dominant bacgkround contribution to the SR comes from the non-resonant
multijet QCD process. Unfortunately the estimation of this process through MC
is not reliable due to the statistical precision of available samples, as well
as underlying inaccuracies of the event generation.  Thus, a data-driven
estimate was employed by fitting the large-$R$ jet mass distribution, $m_{J}$,
in the SR with a parametric function after validation of the procedure in the
$\text{CR}_{\text{QCD}}$.  This approach is further motivated by the good
agreement in shape of the $\text{CR}_{\text{QCD}}$ and SR over the mass range
of the fit, $70~\GeV$ to $230~\GeV$, as seen in \Cref{fig:selection:sr_cr_shape}.

The statistical precision of $\sim 1 \text{fb}^{-1}$ of the
$\text{CR}_{\text{QCD}}$ is found to be comprable to that of the SR for a
luminosity of $80.5 \text{fb}^{-1}$.  Thus, the $\text{CR}_{\text{QCD}}$ was
broken into 60 slices constructed using adjacent data runs resulting in an
average of $1.2 \text{fb}^{-1}$ of data per slice.

Two families of parametric functions were used for this modeling.  The
polynomial exponential function was chosen to be the nominal model

\begin{equation}
\label{sec:background:polynomial}
f_{n} \left(x \middle|\,\vec{\theta}\right) = \theta_{0}\, \exp\left(\sum_{i=1}^{n} \theta_{i}\,x^{i}\right), \quad x = \frac{m_{J} - 150~\GeV}{80~\GeV},
\end{equation}

and the alternative model, the Formal Laurent series,

\begin{equation}
\label{sec:background:laurent}
f_{n} \left(x \middle|\,\vec{\theta}\right) = a \sum_{i=0}^{n} \frac{\theta_{i}}{x^{i+1}}, \quad a=10^{5},\,x = \frac{m_{J} + 90~\GeV}{160~\GeV}.
\end{equation}

The $\theta$ coefficeints are determined by the fit, $a$ is empirically chosen
to keep the scale of parameters at $\mathcal{O}$, and the independent variable
$x$ parameterizes the fit range $m_{J}\in[70,230]$ GeV to $x\in[-1,1]$ for
\Cref{sec:background:polynomial} and $x\in[1,2]$ for \Cref{sec:background:laurent}.
This reparameterization was exmpirically seen to provide improved numerical
stability in the fit.

Both functions are tested on a random $\text{CR}_{\text{QCD}}$ slice to
determine the minimum number of model parameters needed to describe the shape
of the distribution.  The $Z$ + jets, $W$ + jets, and
$k$-factor corrected $t\bar{t}$ contributions are all scaled
by their cross sections times the luminosity and then subtracted from the slice
to remove bias from the fit.  The results of a likelihood ratio test with
Wilks' theorem \cite{wilks1938} and the $F$-test
\cite{snecdecor1991statistical} were both used to determine the minimum number
of model parameters for both function choices.  The Wilks test preferred a
five-parameter model for both while the $F$-test preferred a four-parameter
model for both.  To be conservative in our final estimate, the five-parameter
model was chosen for both the polynomail exponential function and the Formal
Laurent series.

In order to determine the robustness of these fit functions they were validated
using all of the $\text{CR}_{\text{QCD}}$ slices.  For these validation studies
the fit included properly scaled $Z$ + jets, $W$ + jets, $t\bar{t}$ and single
top contributions in addition to the $\text{CR}_{\text{QCD}}$ slice. Within the
statistical precision given by the different data slices, the $\chi^{2}/ndf$
from the individual fits is found to follow the expected distribution of a good
fit.
