\chapter{Background Estimation} \label{chap:background}

Critical to the final statistical interpretation discussed in \Cref{chap:fit}
is the estimated contribution of the resonant and non-resonant backgrounds
considered in this analysis. The resonant $V+\text{jets}$ and $t\bar{t}$
processes are estimated using the MC samples discussed in
\Cref{sec:data:bkg_mc}, whereas a data-driven technique is used to estimate the
non-resonant QCD background. The final fit is done in the invariant mass
spectrum of the signal candidate in the SR over the range $70~\GeV < m_{J} <
230~\GeV$.  As shown in \Cref{simulated_background_shapes} the Higgs signal is
small, and on the tails of the comprably larger resonant backgrounds.  This
means that any statistical fluctuation in the MC templates could hide the
signal.  To avoid this, smooth parametric shapes are fitted to the resonant MC
histograms and then used as the inputs to the fit.  This chapter covers the
creation of the $V+\text{jets}$ template, the $t\bar{t}$ template including the
derivation of a correction to it's normalization using a fit in the
$\text{CR}_{t\bar{t}}$, and the data-driven modeling of the QCD background in
the $\text{CR}_{\text{QCD}}$.

\section{Hadronic Vector Boson Background} \label{sec:background:vqq}

The $V+\text{jets}$ template is constructed by fitting the generated MC
histogram with the sum of three Gaussians plus a constant term.  The systematic
variations, discussed in \Cref{chap:systematics}, of this template are re-fit
using the same functional choices.  This results in smooth histograms to be
used as input to the fit discussed in \Cref{chap:fit}.

\section{t$\overline{t}$ control region} \label{sec:background:ttbar}

\section{Multi-jet QCD estimation} \label{sec:background:qcd}

