\chapter{Background Estimation} \label{chap:background}

The estimated contribution of the resonant and non-resonant backgrounds is
critical to the final statistical interpretation discussed in \Cref{chap:fit} .
The resonant $V+\text{jets}$ and $t\bar{t}$ processes are estimated using the
MC samples discussed in \Cref{sec:data:bkg_mc}, whereas a data-driven technique
is used to estimate the non-resonant QCD background. The final fit is done in
the invariant mass spectrum of the signal candidate in the SR over the range
$70~\GeV < m_{J} < 230~\GeV$.  As shown in \Cref{simulated_background_shapes},
the Higgs boson signal is small, and it is on the tails of the comparably
larger resonant backgrounds.  This means that any statistical fluctuation in
the MC templates could hide the signal.  To avoid this, smooth parametric
shapes are fitted to the resonant MC histograms and then used as the inputs to
the fit.  This chapter covers the creation of the $V+\text{jets}$ template; the
$t\bar{t}$ template, including the derivation of a correction to its
normalization using a fit in the $\text{CR}_{t\bar{t}}$; and the data-driven
modeling of the QCD background in the $\text{CR}_{\text{QCD}}$.

\begin{figure}[!htbp]
  \centering
  \subcaptionbox{Simulation of the contribution of non-resonant multijet QCD background in the signal region.}{\includegraphics[width=0.49\linewidth]{figures/data/QCD_logy}} \hfill
  \subcaptionbox{Simulation of resonant $V$+jets and $t\bar{t}$ backgrounds in the signal region.  The Higgs simulation is included for comparison.}{\includegraphics[width=0.49\linewidth]{figures/data/resonant_bkg}}
  \caption{Simulations of the non-resonant and resonant background contributions to the signal region for an integrated luminosity of $80.5~\ifb$}
  \label{simulated_background_shapes}
\end{figure}


\section{Hadronic Vector Boson Background} \label{sec:background:vqq}

The $V+\text{jets}$ template is constructed by fitting the generated MC
histogram with the sum of three Gaussians plus a constant term.  The systematic
variations, discussed in \Cref{chap:systematics}, of this template are re-fit
using the same functional choices.  This results in smooth histograms to be
used as input to the fit discussed in \Cref{chap:fit}.

\section{t$\overline{t}$ control region} \label{sec:background:ttbar}

\section{Multi-jet QCD estimation} \label{sec:background:qcd}

