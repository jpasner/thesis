\section{LHC layout and design} \label{sec:lhc:layout}

While often depicted as a perfect circle the LHC is in reality an octagon with
rounded edges, called arcs, as can be seen in Figure XXX.  Here you can see the
counter circulating beams of protons depicted in red and blue.  These beams are
focused and collided at the 4 dedicated interaction points at rates of up to 40
MHz.  Two of these points are occupoied by the  ATLAS and CMS experiments, both
of which are high luminosity, multi-purposed experiments.

The exact design of the tunnel is due to the experimental constraints of the
original machine for which it was built, the Large Electron Positron (LEP)
Collider.  For the $\sim 2,000$ times lighter electron the maximum energy was
limited by the synchrotron radiation, proportional to $\frac{1}{m^4}$, requiring
long straight sections of accelerating RF cavities to recouperate the lost
energy.  Given that this effect is $\mathcal{O}(10^{13})$ times smaller for the
proton the LHC is instead limited by our ability to design and construct magnets
strong enough to bend the beam given the already determined curvature of the 8
arcs.

The oppositely rotating beams must each  have their own ring and magnetic field
which lead to the creation of a twin-bore (i.e. "two-in-one") magnet design, a
cross section of which can be seen in Figure XXX. These magnets are constructed
using NbTi superconductors which are cooled to 2K using superfluid helium.
These magnets are designed to provide the needed 8.33 T magnetic field required
to bend the beams at the design beam energy of 7 TeV.  In total 1231 of these 15
m long bending dipole magnets are used, in association with 392 5-7m long
quadrupole magnets which are responsible for keeping the proton bunches in a
tight beam by squeezing them either horizontally or vertically.
