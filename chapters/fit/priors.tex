\section{Implementation of Priors} \label{sec:fit:priors}

As discussed in \Cref{sec:fit:theory} the priors representing the nuisance
parameters and the signal normalization are chosen to represent the analyzer's
knowledge before the data is considered. Generally speaking, the shape of the
prior distribution for systematic uncertainties is chosen to give decreased
probability as the fit tries to pull the value away from their nominal value,
while for the signal parameter the choice is made to have as little influence
on the marginalized posterior as possible.

For the $t\bar{t}$ signal strength a Gaussian prior is used with the mean and
width determined by the fit to data in the $\text{CR}_{t\bar{t}}$.  The
Gaussian prior is them normalized to unity and defined over the range [$-\nu_{5
\cdot t\bar{t}}, \nu_{5 \cdot t\bar{t}}$] where $\nu_{5 \cdot t\bar{t}}$ is five
times the expected value for $t\bar{t}$ and is set to zero elsewhere.

The remaining nuisance parameters are included using a Gaussian prior for each
source.  The Gaussian for the QCD fit function choice uncertainty is defined in
the range [$0\sigma, 1\sigma$], where $0\sigma$ corresponds to the nominal fit
function and $1\sigma$ corresponds to the alternate fit function.  All other
sources of systematic uncertainty are defined using a Gaussian over the large
range $[-3\sigma, 3\sigma]$ to allow ample room for fluctuations.
 
The two signal models, $V$ + jets and Higgs + jets, are eachi included in the
combined fit utilizing a uniform prior to represent the parameter corresponding
to the number of events.  This is done to remove analyzer bias, thus allowing
the final result to more accurately reflect the data.  Furthermore, both priors
are normalized to unity over their respective ranges for reasons discussed in
\Cref{sec:fit:bat}. The $V$ + jets uniform prior is defined to be $1/(2 \cdot
\nu_{5 \cdot V})$ over the range [$-\nu_{5 \cdot V}, \nu_{5 \cdot V}$] and zero
everywhere else. Here $\nu_{5 \cdot V}$ is defined to be five times the expected
value for $V$ + jets. The Higgs + jets uniform prior is defined to be
$1/\nu_{Hmax}$ over the range [$0,\nu_{Hmax}$] and zero elswhere.  This
$\nu_{Hmax}$ is the number of Higgs + jets events corresponding to the point
where the likelihood is a factor of $10^{5}$ times smaller than its maximum
value.  In both cases the ranges where chosen to be very large so as to not
influence the result.
