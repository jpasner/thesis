\section{Uncertanties on Resonant Backgrounds and Signal} \label{sec:systematics:resonant_modeling}

All MC templates in this analysis contain uncertainties derived from the
large-$R$ jet energy and mass calibrations \cite{Aaboud:2018kfi} and the
calibration of the \texttt{MV2c10} $b$-tagging algorithm \cite{Aaboud:2018xwy}
which effects $b$, $c$, and $l$ jet flavors differently.  The jet energy and
jet mass calibration uncertanties affect both normalization and shape of the
templates.  This means their impact on the analysis must be determined by
varying the jet energy and jet mass up and down within their uncertanties and
propogating the varied templates through the entire analysis procedure.  The
effect of the jet energy resolution uncertainty was also tested, but found to
be negligible. The impacts of uncertainties on the calibration of the
\texttt{MV2c10} algorithm have been found to be independent of the large-$R$
jet mass for all MC templats and thus only affect the signal normalization.

To account for the modeling uncertanties associated with the choice of MC
generator, additional shape uncertanties are applied to the $V$ + jets and
$t\bar{t}$ templates.  The systematic is determined by taking the difference
between the large-$R$ jet mass shapes from two different MC generators.  For
$V$ + jets the nominal shape was generated using \textsc{Sherpa} 2.1.1 and then
compared to the alternate shape generated with \textsc{Herwig}++ 2.7.  For
$t\bar{t}$ the nominal shape was generated with \textsc{Powheg-Box} 2 and
compared with the alternate shape generated with \textsc{Sherpa} 2.1.1.

The $t\bar{t}$ normalization in the SR is constrained by the $k$-factor derived
from a fit to data in the $\text{CR}_{t\bar{t}}$ as discussed in
\Cref{sec:background:ttbar}.  The uncertainty on the measurement was found to
be $13\%$ and is treated as a systematic uncertainty on the $t\bar{t}$
normalization.

In order to have an accurate simulation the normalization of all MC templates
must be scaled by the integrated luminosity of the dataset in question.  The
systematic uncertainty on the integrated luminosity measurement is derived
following the methodology presented in Reference \cite{Aad:2013ucp}.  For this
analysis it was found to be $2.1\%$.

Theoretical systematic uncertanties on the normalization of the $V$ + jets and
Higgs components are added to the fit in the SR.  Theory uncertainties on the
$V$ + jets processes result from the impact of higher order electroweak and QCD
corrections to their differential cross sections \cite{Lindert:2017olm}. For
Higgs production the dominant theoretical uncertainty is from ggF production
which is taken to be $30\%$.  This is consistent with the cross section
uncertainty calculated with the MiNLO procedure and includes the effects of the
non-zero top-mass for Higgs bosons with $p_{T} > 400~\GeV$.  This same
uncertainty is applied for the other production mechanisms for Higgs' with
$p_{T} > 400~\GeV$.  The result is a total theory uncertainty on the Higgs
cross section of $30\%$.
