\section{Summary of Systematics} \label{sec:systematics:summary}

The signal strength ($\mu_{s}$) is defined as the scale factor which multiplies
the cross section times branching fraction that is predicted by the signal
hypothesis being considered.  Thus, the final fit, discussed in
\Cref{chap:fit}, of our theory based MC to the SR data provides a direct
measurement of this signal strength and therefore a direct measurement of the
model's prediction given the data. However, to understand the significance and
error on this result the statistical and systematic uncertainties must be taken
into account during the fitting procedure.

To get a sense of the individual impact of each systematic on the total
uncertainty for the measurement of $\mu_{s}$, the following ad hoc comparison
is made.  First, the total uncertainty on $\mu_{s}$ ($\sigma_{tot}$) is
determined by running the fit with all sources of error (systematic and
statistical) allowed to float within their defined constraints.  By definition:

$$ \sigma_{\text{tot}}^{2} \; = {\underbrace{\sum_{i}^{n} \sigma_{i}^{2}}_{\substack{\text{Contributions from} \\ \text{n systematics}}}} + {\underbrace{\vphantom{\sum_{i}^{n}} \sigma_{\text{stat.}}}_{\substack{\text{Contribution from} \\ \text{statistics}}}}$$

Next, the fit is re-run with all systematics fixed to their pre-fit values
except for the systematic being investigated for its impact. This gives us
$\sigma_{i}$, the uncertainty on $\mu_{s}$ when only considering the effect of
the $i^{th}$ nuisance parameter. Note that the statistical uncertainty
($\sigma_{\text{stat}}$) is still included as it is inherent to the fitting
procedure.

Finally, the difference in quadrature between $\sigma_{\text{tot}}$ and
$\sigma_{i}$ is found and divided by the measured $\mu_{s}$ value found in the
final fit which includes all sources of uncertainty. 

$$ \frac{\sqrt{\Delta \sigma_i^2}}{\mu_{s}} = \frac{\sqrt{\sigma_{tot}^2 - \sigma_i^2}}{\mu_{s}} $$

This ad hoc comparison represents the impact of the $i^{th}$ nuisance parameter
on the total uncertainty of the measurement of $\mu_{s}$. A summary of the
impact of each systematic for the two considered signal models (Higgs + jets
and $V$ + jets) is presented in \Cref{table:systematic_uncertainties}

\begin{table}[htpb]
 \centering
 \caption{
  Summary of the impact $(\sqrt{\Delta \sigma_i^2}/\mu_{s})$ of the main systematic uncertainties on the total uncertainty, $\sigma_{tot}$, of the measured signal strength, $\mu_{s}$, for the $V$ + jets and Higgs boson + jets signals signals~\cite{ATLAS-CONF-2018-052}.}
 \begin{adjustbox}{max width=\textwidth}
  \begin{tabular}{@{}llrrrr@{}}
   \toprule
   Source                    & Type           & $V$ + jets & Higgs + jets  \\ \midrule
   Jet energy and mass scale & Norm. \& Shape & $15\%$     & $14\%$ \\
   Jet mass resolution       & Norm. \& Shape & $20\%$     & $17\%$ \\
   $V$ + jets modeling       & Shape          & $9\%$      & $4\%$  \\
   $t\bar{t}$ modeling       & Shape          & $<1\%$     & $1\%$  \\
   $b$-tagging $(b)$         & Normalization  & $11\%$     & $12\%$ \\
   $b$-tagging $(c)$         & Normalization  & $3\%$      & $1\%$  \\
   $b$-tagging $(l)$         & Normalization  & $4\%$      & $1\%$  \\
   $t\bar{t}$ $k$-factor     & Normalization  & $2\%$      & $3\%$  \\
   Luminosity                & Normalization  & $2\%$      & $2\%$  \\
   Alternate QCD function    & Norm. \& Shape & $4\%$      & $4\%$  \\
   $W$/$Z$ and QCD (Theory)  & Normalization  & $14\%$     &        \\
   Higgs (Theory)            & Normalization  &            & $30\%$ \\
   \bottomrule
  \end{tabular}
 \end{adjustbox}
 \label{table:systematic_uncertainties}
\end{table}
