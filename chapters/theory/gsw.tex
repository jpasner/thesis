\section{Electroweak Unification} \label{sec:theory:gsw}

In the SM the electromagnetic and weak forces are unified by GSW theory into
the Electroweak interaction which is represented by the $SU(2)_L \times U(1)_Y$
gauge group \cite{\cite{Glashow:1961tr, Goldstone:1962es, Weinberg:1967tq}}.
The $L$ represents the experimental observation that the weak interaction, and
thus the $SU(2)$ transformation, only acts on left-handed particle states.  The
$Y$ indicates that this is the $U(1)$ symmetry for the weak hypercharge $Y$
instead of the electromagnetic charge.  The particle states for these
interactions are solutions to the Dirac equation and are represented as isospin
doublets ($\boldsymbol{\Psi}_{L}$) for the left-handed states, and as isospin
singlets ($\Psi_R$) for the right-handed states.  Thus a general transformation
from the electroweak gauge group applied to the left-handed doublet is
represented as

\begin{equation} \label{eq:gsw:doublet} 
\boldsymbol{\Psi}_{L} \rightarrow \boldsymbol{\Psi}_{L}^{'} = \exp \left(
\underbrace{ ig^{'} \frac{Y_L}{2}
\zeta (x) }_{U(1)_Y} + \underbrace{ ig_{W} \boldsymbol{\alpha}(x) \cdot
\text{\bf{T}} }_{SU(2)_L} \right) \boldsymbol{\Psi}_{L}.
\end{equation}

For the right-handed singlet the $SU(2)_L$ transformation does not contribute,
so the transformation is

\begin{equation} \label{eq:gsw:singlet} 
{\Psi_R} \rightarrow \Psi_R^{'} = \exp \left( \underbrace{ ig^{'} \frac{Y_R}{2}
\zeta (x) }_{U(1)_Y} \right) \Psi_R.
\end{equation}

Here the local gauge transformations have introduced space-time dependent terms
$\boldsymbol{\alpha}(x)$ and $\zeta(x)$ into the electroweak Lagrangian.  Due
to the derivatives contained within the kinetic term of this Lagrangian, this
new configuration would introduce additional terms, thus violating the required
local gauge invariance.  These additional terms can be removed by replacing the
standard derivative ($\partial_{\mu}$) with the covariant derivative
($D_{\mu}$), as seen in \Cref{eq:gsw:left} for the left-handed states and
\Cref{eq:gsw:right} for the right-handed states.

\begin{align} 
\label{eq:gsw:left} 
D_\mu &= \partial_{\mu}
- \underbrace{\frac{1}{2}ig^{'}B_{\mu}Y_{L}}_{U(1)_Y} -
  \underbrace{\frac{1}{2}ig_{W}\bf{W}_\mu \cdot \boldsymbol{\tau}}_{SU(2)_L} \\
\label{eq:gsw:right} 
D_\mu &= \partial_{\mu}  - \underbrace{\frac{1}{2}ig^{'}B_{\mu}Y_{R}}_{U(1)_Y} 
\end{align}

This introduces two new gauge fields: the weak hypercharge field $B_\mu$ and
the charged weak fields $\boldsymbol{W_\mu}$ as well as the associated coupling
constants $g^{'}, g_{W}, Y_{L}, Y_{R}$ and the $SU(2)$ generators
$\boldsymbol{\tau}$.  The transformation properties of these new fields are
given by

\begin{align}
\boldsymbol{W}_{\mu}(x) &\rightarrow \boldsymbol{W}_{\mu}^{'}(x) =
\boldsymbol{W}_{\mu} + \partial_{\mu} \boldsymbol{\alpha}(x) +
g_{W}\boldsymbol{W}_{\mu}(x) \times \boldsymbol{\alpha}(x)
\\ 
B_{\mu} &\rightarrow B_{\mu}^{'} = B_{\mu} +
\frac{1}{g^{'}}\partial_{\mu}\zeta(x)
\end{align}

The form of these fields is chosen such that the final Lagrangian is invariant
under $SU(2)_L \times U(1)_Y$ transformations, and thus gauge invariance has
been restored for the kinetic term of the electroweak Lagrangian.  Inserting
these new definitions into the Lagrangian for the field $\Psi$ that satisfies
the free-particle Dirac equation gives

\begin{equation} \label{eq:gsw:fermion_lagrangian}
\mathcal{L} = i\boldsymbol{\bar{\Psi}}_{L}\gamma^{\mu} \left( \partial_{\mu}
- \frac{1}{2}ig^{'}B_{\mu}Y_{L} - \frac{1}{2}ig_{W}\bf{W}_\mu \cdot
  \boldsymbol{\tau} \right) \boldsymbol{\Psi}_{L} + i \bar{\Phi}_{R}\gamma^{\mu}
\left(\partial_{\mu} - \frac{1}{2}ig^{'}B_{\mu}Y_{R} \right) \Phi_{R}
\end{equation}

Next the gauge field self-interaction and mass terms are constructed

\begin{equation} \label{eq:gsw:gauge_lagrangian}
\mathcal{L} = -\frac{1}{4}\boldsymbol{F}_{\mu\nu}\boldsymbol{F}^{\mu\nu}
-\frac{1}{4}B_{\mu\nu}B^{\mu\nu} +
\frac{1}{2}M_{W}^{2}\boldsymbol{W}_{\mu}\boldsymbol{W}^{\mu} +
\frac{1}{2}M_{B}^{2}B_{\mu}B^{\mu}
\end{equation}

where the field tensors $\boldsymbol{F}^{\mu\nu}$ and $B^{\mu\nu}$ are defined
to be

\begin{align}
\boldsymbol{F}^{\mu\nu}  &= \partial^{\mu}\boldsymbol{W}^{\nu} -
\partial^{\nu}\boldsymbol{W}^{\mu} + g\boldsymbol{W}^{\mu} \times
\boldsymbol{W}^{\nu} \\
B^{\mu\nu} &=  \partial^{\mu}\boldsymbol{B}^{\nu} -
\partial^{\nu}\boldsymbol{B}^{\mu}
\end{align}

The field tensor terms in \Cref{eq:gsw:gauge_lagrangian} are invariant under
gauge transformations, but simply substituting in \Cref{eq:gsw:left} or
\Cref{eq:gsw:right} into the mass terms shows that these terms violate gauge
invariance. This would imply that $M_{W} = 0$ and $M_{B} = 0$, in direct
contradiction of the observed masses of the weak gauge bosons.  This issue
arises again for fermion mass terms, as illustrated below for the electron field
($e$) expanded in its chiral basis.

\begin{equation}
m_{e}\bar{e}e = m_{e} \left( \begin{matrix}e^{\dagger}_{R} &
e^{\dagger}_{L} \end{matrix} \right) \left( \begin{matrix} e_{L}
\\ e_{R} \end{matrix} \right) = m_{e}(e^{\dagger}_{R}e_{L} +
e^{\dagger}_{L}e_{R})
\end{equation}

Remembering that the left- and right-handed components of the electroweak
interaction transform differently shows that this mixture of right- and left-
handed fields violates gauge invariance. This again forces us to conclude that
$m_{e} = 0$, in contradiction to the observation that the electron does indeed
have mass. As mentioned in \Cref{sec:theory:bosons} the resolution to these
mass mysteries lies in the Higgs mechanism discussed in \Cref{sec:theory:higgs}.
