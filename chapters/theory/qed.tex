\section{Quantum Electrodynamics} \label{sec:theory:qed}

In the SM the Electromagnetic and Weak nuclear forces are unified into the
Electroweak interaction which is represented by the $SU(2)_L \times U(1)_Y$
gauge group. The L represents the physical observable that the Weak interaction,
and thus the $SU(2)$ transformation, only acts on left handed particle states.
The Y states that this is the $U(1)$ symmetry for the weak hypercharge Y instead
of the electromagnetic charge.  The particle states for these interactions are
solutions to the Dirac equation and are represented as Dirac spinor doublets
($\boldsymbol{\Psi_L}$) for the left handed states, and as Dirac spinor singlets
($\Phi_R$) for the right handed states.  Thus when a general transformation from
the Electroweak gague group is applied to the left handed spinor doublet you get
equation \ref{eq:qed:doublet}

\begin{equation} \label{eq:qed:doublet} 
\boldsymbol{\Psi_L} \rightarrow \boldsymbol{\Psi_L}^{'} = exp \left(
\underbrace{ ig^{'} \frac{Y_L}{2}
\zeta (x) }_{U(1)_Y} + \underbrace{ ig_{W} \boldsymbol{\alpha}(x) \cdot
\text{\bf{T}} }_{SU(2)_L} \right) \boldsymbol{\Psi_L}.
\end{equation}

For the right handed spinor singlet the $SU(2)_L$ doesn't contribute and
you get equation \ref{eq:qed:singlet}

\begin{equation} \label{eq:qed:singlet} 
{\Phi_R} \rightarrow \Phi_R^{'} = exp \left( \underbrace{ ig^{'} \frac{Y_R}{2}
\zeta (x) }_{U(1)_Y} \right) \Phi_R.
\end{equation}

We can see that these local gauge transformations have introduced space-time
depended terms $\boldsymbol{\alpha}$ and $\zeta$ into our electroweak
Lagrangian.  Due to the derivatives contained within the kinetic term of this
lagrangian, this new configuration would introduce additional terms, thus
violating our required local gauge invariance.  Luckily, we can remove these
additional terms by replacing the standard derivative ($\partial_{\mu}$) with th
covariant derivative ($D_{\mu}$) as seen in equation \ref{eq:qed:left} for the
left handed states and \ref{eq:qed:right} for the right handed states.

\begin{equation} \label{eq:qed:left} 
D_\mu = \partial_{\mu}
- \underbrace{\frac{1}{2}ig^{'}B_{\mu}Y_{L}}_{U(1)_Y} -
  \underbrace{\frac{1}{2}ig_{W}\bf{W}_\mu \cdot \boldsymbol{\tau}}_{SU(2)_L}
\end{equation}

\begin{equation} \label{eq:qed:right} 
D_\mu = \partial_{\mu}  - \underbrace{\frac{1}{2}ig^{'}B_{\mu}Y_{R}}_{U(1)_Y} 
\end{equation}

Here we see two new gauge fields; $B_\mu$ the weak hypercharge field and
$\boldsymbol{W_\mu}$ the charged weak field.  The form of these fields is chosen
such that the final Lagrangian is invariant under $SU(2)_L \times U(1)_Y$
transformations, and thus we have restored gauge invariance for the kinetic term
of our electroweak Lagrangian!  Inserting these new definitions into the
Lagrangian for the spinor field $\Psi$ which satisfies the free-particle Dirac
equation we get

\begin{equation} \label{eq:qed:lagrangian}
\mathcal{L} = i\bar{\boldsymbol{\Psi_L}}\gamma^{\mu} \left( \partial_{\mu}
- \frac{1}{2}ig^{'}B_{\mu}Y_{L} - \frac{1}{2}ig_{W}\bf{W}_\mu \cdot
  \boldsymbol{\tau} \right) \boldsymbol{\Psi_L} + i \Phi_{R}\gamma^{\mu}
\left(\partial_{\mu} - \frac{1}{2}ig^{'}B_{\mu}Y_{R} \right) \Phi_{R}
\end{equation}

Wavefunction
Transformation
Covariant Derivative
Lagrangian
Identify Terms

In the SM the electromagnetic and weak nuclear forces are unified into the
electroweak interaction which . . . 

Quantum Electrodynamics is the first model created in the QFT image.
