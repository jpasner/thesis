\section{Quantum Electrodynamics} \label{sec:theory:qed}

In the SM the Electromagnetic and Weak nuclear forces are unified into the
Electroweak interaction which is represented by the $SU(2)_L \times U(1)_Y$
gauge group. The L represents the physical observable that the Weak interaction,
and thus the $SU(2)$ transformation, only acts on left handed particle states.
The Y states that this is the $U(1)$ symmetry for the weak hypercharge Y instead
of the electromagnetic charge.  The particle states for these interactions are
solutions to the Dirac equation and are represented as Dirac spinor doublets
($\boldsymbol{\Psi_L}$) for the left handed states, and as Dirac spinor singlets
($\Psi_R$) for the right handed states.  Thus when a general transformation from
the Electroweak gague group is applied to the left handed spinor doublet you get
\cref{eq:qed:doublet}

\begin{equation} \label{eq:qed:doublet} 
\boldsymbol{\Psi_L} \rightarrow \boldsymbol{\Psi_L}^{'} = exp \left(
\underbrace{ ig^{'} \frac{Y_L}{2}
\zeta (x) }_{U(1)_Y} + \underbrace{ ig_{W} \boldsymbol{\alpha}(x) \cdot
\text{\bf{T}} }_{SU(2)_L} \right) \boldsymbol{\Psi_L}.
\end{equation}

For the right handed spinor singlet the $SU(2)_L$ doesn't contribute and
you get \cref{eq:qed:singlet}

\begin{equation} \label{eq:qed:singlet} 
{\Psi_R} \rightarrow \Psi_R^{'} = exp \left( \underbrace{ ig^{'} \frac{Y_R}{2}
\zeta (x) }_{U(1)_Y} \right) \Psi_R.
\end{equation}

We can see that these local gauge transformations have introduced space-time
dependant terms $\boldsymbol{\alpha}(x)$ and $\zeta(x)$ into our electroweak
Lagrangian.  Due to the derivatives contained within the kinetic term of this
lagrangian, this new configuration would introduce additional terms, thus
violating our required local gauge invariance.  Luckily, we can remove these
additional terms by replacing the standard derivative ($\partial_{\mu}$) with th
covariant derivative ($D_{\mu}$) as seen in \cref{eq:qed:left} for the
left handed states and \cref{eq:qed:right} for the right handed states.

\begin{align} 
\label{eq:qed:left} 
D_\mu &= \partial_{\mu}
- \underbrace{\frac{1}{2}ig^{'}B_{\mu}Y_{L}}_{U(1)_Y} -
  \underbrace{\frac{1}{2}ig_{W}\bf{W}_\mu \cdot \boldsymbol{\tau}}_{SU(2)_L} \\
\label{eq:qed:right} 
D_\mu &= \partial_{\mu}  - \underbrace{\frac{1}{2}ig^{'}B_{\mu}Y_{R}}_{U(1)_Y} 
\end{align}

Here we see two new gauge fields; $B_\mu$ the weak hypercharge field and
$\boldsymbol{W_\mu}$ the charged weak field as well as the associated coupling
constants $g^{'}, g_{W}, Y_{L}, Y_{R}$ and the $SU(2)$ generators
$\boldsymbol{\tau}$.   Next we right down the transformation properies of these
new fields 

\begin{align}
\boldsymbol{W}_{\mu}(x) &\rightarrow \boldsymbol{W}_{\mu}^{'}(x) =
\boldsymbol{W}_{\mu} + \partial_{\mu} \boldsymbol{\alpha}(x) +
g_{W}\boldsymbol{W}_{\mu}(x) \times \boldsymbol{\alpha}(x)
\\ 
B_{\mu} &\rightarrow B_{\mu}^{'} = B_{\mu} +
\frac{1}{g^{'}}\partial_{\mu}\zeta(x)
\end{align}

The form of these fields is choosen such that the final Lagrangian is invariant
under $SU(2)_L \times U(1)_Y$ transformations, and thus we have restored gauge
invariance for the kinetic term of our electroweak Lagrangian!  Inserting these
new definitions into the Lagrangian for the spinor field $\Psi$ which satisfies
the free-particle Dirac equation we get

\begin{equation} \label{eq:qed:fermion_lagrangian}
\mathcal{L} = i\boldsymbol{\bar{\Psi}_L}\gamma^{\mu} \left( \partial_{\mu}
- \frac{1}{2}ig^{'}B_{\mu}Y_{L} - \frac{1}{2}ig_{W}\bf{W}_\mu \cdot
  \boldsymbol{\tau} \right) \boldsymbol{\Psi_L} + i \bar{\Phi}_{R}\gamma^{\mu}
\left(\partial_{\mu} - \frac{1}{2}ig^{'}B_{\mu}Y_{R} \right) \Phi_{R}
\end{equation}

Next we must construct the gauge field self interaction and mass terms

\begin{equation} \label{eq:qed:gauge_lagrangian}
\mathcal{L} = -\frac{1}{4}\boldsymbol{F}_{\mu\nu}\boldsymbol{F}^{\mu\nu}
-\frac{1}{4}B_{\mu\nu}B^{\mu\nu} +
\frac{1}{2}M_{W}^{2}\boldsymbol{W}_{\mu}\boldsymbol{W}^{\mu} +
\frac{1}{2}M_{B}^{2}B_{\mu}B^{\mu}
\end{equation}

where the field tensors $\boldsymbol{F}^{\mu\nu}$ and $B^{\mu\nu}$ are defined
to be

\begin{align}
\boldsymbol{F}^{\mu\nu}  &= \partial^{\mu}\boldsymbol{W}^{\nu} -
\partial^{\nu}\boldsymbol{W}^{\mu} + g\boldsymbol{W}^{\mu} \times
\boldsymbol{W}^{\nu} \\
B^{\mu\nu} &=  \partial^{\mu}\boldsymbol{B}^{\nu} -
\partial^{\nu}\boldsymbol{B}^{\mu}
\end{align}

The field tensor terms in \cref{eq:qed:gauge_lagrangian} are invariant
under our gauge transformations, but simply plugging in
\cref{eq:qed:left} or \cref{eq:qed:right} into the mass terms shows that
these terms violate gauge invariance thus implying $M_{W} = 0$ and $M_{B} = 0$
in direct contradiction of the observed masses of the weak gauge bosons.  This
issue arises again for fermion mass terms as illustrated below for the elctron
field ($e$) expanded in its chiral basis.

\begin{equation}
m_{e}\bar{e}e = m_{e} \left( \begin{matrix}e^{\dagger}_{R} &
e^{\dagger}_{L} \end{matrix} \right) \left( \begin{matrix} e_{L}
\\ e_{R} \end{matrix} \right) = m_{e}(e^{\dagger}_{R}e_{L} +
e^{\dagger}_{L}e_{R})
\end{equation}

Remembering that the left and right handed spinors of the electroweak
interaction transform differently we see that this mixture of right and left
fields violates gauge invariance. This again forces us to conclude that $m_{e} = 0$
in contradiction to the observation that the electron does indeed have mass. As
mentioned in \cref{sec:theory:bosons} the resolution to these mass
mysteries lies in the Higgs mechanism discussed in
\cref{sec:theory:higgs}
