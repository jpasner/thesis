\section{Quantum Chromodynamics} \label{sec:theory:qcd}

Quantum Chromodynamics is the continuation of the mathematical framework
established by the electroweak formalism in (\Cref{sec:theory:gsw}, this time
for the strong force described by the $SU(3)_C$ gauge group, where $C$
represents the ``color" charge of QCD \cite{\cite{Campbell:2017hsr}}.  This
color charge doesn't imply actual visible color, but is useful as an analogy to
the visible spectrum where a combination of red, green, and blue generates
white.  For QCD the combination of red, green, and blue color charges can
result in a colorless object.  As mentioned in \Cref{sec:theory:fermions}, the
quarks have a color (anti-color) charge defining color triplet field which
transforms under the general $SU(3)$ transformation as

\begin{equation}
q = \left( \begin{matrix} q_{r} \\ q_{g} \\ q_{b} \end{matrix} \right)
\rightarrow q^{'} = exp \left( ig_{s} \sum_{k=1}^{8} \eta_{k}(x)
\frac{\lambda_k}{2} \right) q
\end{equation}

Here the $\lambda_{k}$ are the Gell-Mann generators for $SU(3)$, $\eta(x)_{k}$ is the
space-time dependency for each generator, and  $g_s$ is the strong coupling constant.
As with the electroweak Lagrangian, the introduction of these space-time dependent terms adds new
terms into the kinematic portion of the Lagrangian and spoils the gauge 
invariance.  Again, a covariant derivative is introduced 

\begin{equation}
D_{\mu} = \partial_{\mu} - ig_{s}G_{\mu}^{k}\frac{\lambda_{k}}{2}
\end{equation}

to restore gauge invariance. The $G_{\mu}^{k}$ are the new fields introduced
for the 8 gluons.  These new fields transform under $SU(3)$ as

\begin{equation} \label{eq:qcd:gluon_field}
G_{\mu}^{k} \rightarrow G_{\mu}^{'k} = G_{\mu}^{k} + \partial_{\mu}\eta_{k}(x) +
g_{s}f_{klm}\eta_{l}(x)G_{\mu}^{m}
\end{equation}

Given these definitions the QCD Lagrangian ($\mathcal{L}_{QCD}$) can be
constructed as 

\begin{equation} \label{eq:qcd:qcd_lagrangian}
\mathcal{L}_{QCD} = \bar{q}(i\gamma_{\mu}D^{\mu} - m_{q})q -
\frac{1}{4}G_{k}^{\mu\nu}G_{k\mu\nu}
\end{equation}

where the gluon field tensor $G_{k}^{\mu\nu}$ is defined as

\begin{equation} \label{eq:qcd:gluon_tensor}
G_{k}^{\mu\nu} = \partial^{\mu}G_{k}^{\nu} - \partial^{\nu}G_{k}^{\mu} +
g_{s}f_{klm}G_{\l}^{\mu}G_{m}^{\nu}.
\end{equation}

The strong force is peculiar in that experiments observe only colorless
objects in the form of bound states of quarks known as hadrons.  Qualitatively,
when the potential between a bound state of quarks (meson or baryon) gets
stronger with separation, unlike the other forces.  At the point where the
system would separate into color-charged objects, it becomes energetically
favorable to produce a quark/anti-quark pair in a process known as
hadronization.  In other words, attempting to separate a bound quark state into
its colored constituents simply results in new colorless bound states.  This
requirement of colorless objects by the strong force is known as color
confinement. For highly energetic strong interactions at hadron colliders the
result is an expanding chain of hadronizing quarks and gluons and their decay
products known as a jet.

