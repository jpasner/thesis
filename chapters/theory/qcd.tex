\section{Quantum Chromodynamics} \label{sec:theory:qcd}

Quantum Chromodynamics is the continuation of the mathematical framework
established by Quantum Electrodynamics (section \ref{sec:theory:qed}, this time
for the strong force described by the $SU(3)_C$ gauge group where the $C$
represents the "color" charge of QCD.  This color charge doesn't imply actual
visible color, but is useful as an anology to the visible spectrum where a
combination of red, green, and blue generates white.  For QCD the combination of
red, green, and blue color charges results in a colorless object.  As mentioned
in section \ref{sec:theory:fermions} the quarks will contain a color
(anti-color) charge represented by a color triplet field which transforms under
the general $SU(3)$ transformation as shown here

\begin{equation}
q = \left( \begin{matrix} q_{r} \\ q_{g} \\ q_{b} \end{matrix} \right)
\rightarrow q^{'} = exp \left( ig_{s} \sum_{k=1}^{8} \eta_{k}(x)
\frac{\lambda_k}{2} \right) q
\end{equation}

Here the $\lambda_{k}$ are the generators for $SU(3)$, $\eta(x)_{k}$ is the
space-time dependancy for each generator, and  $g_s$ is the strong coupling constant.
As with QED, the introduction of these space-time dependant terms introduces new
terms into the kinematic portion of the lagrangian thus spoiling our gauge
invairance.  Again, we introduce a covariant derivative to restore invariance

\begin{equation}
D_{\mu} = \partial_{\mu} - ig_{s}G_{\mu}^{k}\frac{\lambda_{k}}{2}
\end{equation}

Here the $G_{\mu}^{k}$ are the new fields introduced for the 8 gluons.  These
new fields transform under $SU(3)$ as shown in equation
\ref{eq:qcd:gluon_field}

\begin{equation}
G_{\mu}^{k} \rightarrow G_{\mu}^{'k} = G_{\mu}^{k} + \partial_{\mu}\eta_{k}(x) +
g_{s}f_{klm}\eta_{l}(x)G_{\mu}^{m}
\end{equation}

Given these definitions we can construct the QCD Lagrangian
($\mathcal{L}_{QCD}$) as shown in equation \ref{eq:qcd:qcd_lagrangian} where the
gluon field tensor $G_{k}^{\mu\nu}$ is the one defined in equation
\ref{eq:qcd:gluon_tensor}

\begin{equation} \label{eq:qcd:qcd_lagrangian}
\mathcal{L}_{QCD} = \bar{q}(i\gamma_{\mu}D^{\mu} - m_{q})q -
\frac{1}{4}G_{k}^{\mu\nu}G_{k\mu\nu}
\end{equation}

\begin{equation} \label{eq:qcd:gluon_tensor}
G_{k}^{\mu\nu} = \partial^{\mu}G_{k}^{\nu} - \partial^{\nu}G_{k}^{\mu} +
g_{s}f_{klm}G_{\l}^{\mu}G_{m}^{\nu}
\end{equation}

The strong force is peculiar in that we experimentally observe only colorless
objects in the form of bound states of quarks known as hadrons.  Qualitatively,
when a bound state of quarks (meson or baryon) is given sufficeint energy to
separate the strong force dramatically increases in strength.  At the point
where the objects would separate, and thus no longer be colorless, it becomes
energetically favorable to produce a quark/anti-quark pair.  In other words,
attempting to separate a bound quark state into its colored constituents simply
results in new colorless bound states.  This requirement of colorless objects by
the strong force is known as color confinement. 

