\chapter{The Standard Model and Beyond} \label{chap:standard_model}

The Standard Model (SM) of Particle Physics is humanities best "guess" at the
force laws that describe the observed behavior of all particles in our
universe. Its formulation is a collection of Quantum Field Theories (QFT) that
describe the following interactions of elementary matter in Nature: the
electromagnetic force, the weak nuclear force and the strong nuclear force.
Gravity is noticeably absent as currently there is no viable quantum theory for
observed gravitational effects.  The Glashow-Weinberg-Salam theory of Quantum
Electrodynamics (QED) describes the electromagnetic and weak forces, while
Quantum Chromodyanmics (QCD) describes the strong force.  These theories form
the following symmetry group of the Standard Model.

\begin{equation} \label{eq:standardmodel:symmetry_group}
  \underbrace{\text{SU}_\text{C}(3)}_\text{QCD} \otimes \underbrace{\text{SU}_\text{L}(2) \otimes \text{U}_\text{Y}(1)}_\text{QED}.
\end{equation}

The gauge principle states that the SM Lagrangian and its predictions must be
invariant under local transformations using an operator from any of these
constituent groups.  Thus, any theory must only include transformations and
terms that maintain the local invariance of the complete Lagrangian.  In
particular, this requirement was violated by any attempt to include an explicit
mass term for the Gauge Bososns of QED and for all fermions.  Around 1960 a
possible solution to this lack of mass was proposed in the form of the
spontaneous breaking of the ElectroWeak symmetry, now known as the Higgs
mechanism.  In the following sections I will go into more detail about the
Lagrangian formalism of the Standard Model, QCD, QED and this recently verified
Higgs Mechanism.

\section{The Standard Model} \label{sec:theory:standardmodel}

At the turn of the 20th century our understaning of the constituent matter of
the universe was limited to what we could see with microscopes and imply from
the observations of light and elctricity, giving us evidence for both the photon
and the electron.  In the first half of the century we discovered the field of
subatomic physics with Rutherfords 1911 gold foil scattering experiment, and
Dirac successfully demonstrated the quantization of the electromagnetic field,
the first step towards a fully Gauge Invariant Quantum Field Theory.
In the second half we literally delved deeper, discovering that the nucleus
contained structure and extended our theories to include the the complex
mechanics of quarks and gluons.  With the discovery of the Higgs in 2013 the
Standard Model has become an irrefutable framework as can be seen in the high
level of agreement betwee theory experiment in figure \ref{fig:comparison}.

\begin{figure}[!htbp]
  \begin{center}
    \includegraphics[width=0.8\linewidth]{figures/atlas/muon_system}
    \caption{ \cite{PERF-2007-01} A cut-away diagram of the ATLAS muon system
and its many sub-detectors.}
    \label{fig:muon_system}
  \end{center}
\end{figure}

The QCD and QED theories predict two classes of particles: fermions and
bosons. These particles represent the quanta of the quantum fields of the
Standard Model and and the mediators of the fundamental forces of Nature.


\section{Quantum Electrodynamics} \label{sec:theory:qed}

In the SM the Electromagnetic and Weak nuclear forces are unified into the
Electroweak interaction which is represented by the $SU(2)_L \times U(1)_Y$
gauge group. The L represents the physical observable that the Weak interaction,
and thus the $SU(2)$ transformation, only acts on left handed particle states.
The Y states that this is the $U(1)$ symmetry for the weak hypercharge Y instead
of the electromagnetic charge.  The particle states for these interactions are
solutions to the Dirac equation and are represented as Dirac spinor doublets
($\boldsymbol{\Psi_L}$) for the left handed states, and as Dirac spinor singlets
($\Phi_R$) for the right handed states.  Thus when a general transformation from
the Electroweak gague group is applied to the left handed spinor doublet you get
equation \ref{eq:qed:doublet}

\begin{equation} \label{eq:qed:doublet} 
\boldsymbol{\Psi_L} \rightarrow \boldsymbol{\Psi_L}^{'} = exp \left(
\underbrace{ ig^{'} \frac{Y_L}{2}
\zeta (x) }_{U(1)_Y} + \underbrace{ ig_{W} \boldsymbol{\alpha}(x) \cdot
\text{\bf{T}} }_{SU(2)_L} \right) \boldsymbol{\Psi_L}.
\end{equation}

For the right handed spinor singlet the $SU(2)_L$ doesn't contribute and
you get equation \ref{eq:qed:singlet}

\begin{equation} \label{eq:qed:singlet} 
{\Phi_R} \rightarrow \Phi_R^{'} = exp \left( \underbrace{ ig^{'} \frac{Y_R}{2}
\zeta (x) }_{U(1)_Y} \right) \Phi_R.
\end{equation}

We can see that these local gauge transformations have introduced space-time
depended terms $\boldsymbol{\alpha}$ and $\zeta$ into our electroweak
Lagrangian.  Due to the derivatives contained within the kinetic term of this
lagrangian, this new configuration would introduce additional terms, thus
violating our required local gauge invariance.  Luckily, we can remove these
additional terms by replacing the standard derivative ($\partial_{\mu}$) with th
covariant derivative ($D_{\mu}$) as seen in equation \ref{eq:qed:left} for the
left handed states and \ref{eq:qed:right} for the right handed states.

\begin{equation} \label{eq:qed:left} 
D_\mu = \partial_{\mu}
- \underbrace{\frac{1}{2}ig^{'}B_{\mu}Y_{L}}_{U(1)_Y} -
  \underbrace{\frac{1}{2}ig_{W}\bf{W}_\mu \cdot \boldsymbol{\tau}}_{SU(2)_L}
\end{equation}

\begin{equation} \label{eq:qed:right} 
D_\mu = \partial_{\mu}  - \underbrace{\frac{1}{2}ig^{'}B_{\mu}Y_{R}}_{U(1)_Y} 
\end{equation}

Here we see two new gauge fields; $B_\mu$ the weak hypercharge field and
$\boldsymbol{W_\mu}$ the charged weak field.  The form of these fields is chosen
such that the final Lagrangian is invariant under $SU(2)_L \times U(1)_Y$
transformations, and thus we have restored gauge invariance for the kinetic term
of our electroweak Lagrangian!  Inserting these new definitions into the
Lagrangian for the spinor field $\Psi$ which satisfies the free-particle Dirac
equation we get

\begin{equation} \label{eq:qed:lagrangian}
\mathcal{L} = i\bar{\boldsymbol{\Psi_L}}\gamma^{\mu} \left( \partial_{\mu}
- \frac{1}{2}ig^{'}B_{\mu}Y_{L} - \frac{1}{2}ig_{W}\bf{W}_\mu \cdot
  \boldsymbol{\tau} \right) \boldsymbol{\Psi_L} + i \Phi_{R}\gamma^{\mu}
\left(\partial_{\mu} - \frac{1}{2}ig^{'}B_{\mu}Y_{R} \right) \Phi_{R}
\end{equation}

Wavefunction
Transformation
Covariant Derivative
Lagrangian
Identify Terms

In the SM the electromagnetic and weak nuclear forces are unified into the
electroweak interaction which . . . 

Quantum Electrodynamics is the first model created in the QFT image.

\section{Quantum Chromodynamics} \label{sec:theory:qcd}

Quantum Chromodynamics is the continuation of the mathematical framework
established by the electroweak formalism in \Cref{sec:theory:gsw}, this time
for the strong force described by the $SU(3)_C$ gauge group, where $C$
represents the ``color" charge of QCD \cite{Campbell:2017hsr}.  This
color charge doesn't imply actual visible color, but is useful as an analogy to
the visible spectrum where a combination of red, green, and blue generates
white.  For QCD the combination of red, green, and blue color charges can
result in a colorless object.  As mentioned in \Cref{sec:theory:fermions}, the
quarks have a color (anti-color) charge defining color triplet field which
transforms under the general $SU(3)$ transformation as

\begin{equation}
q = \left( \begin{matrix} q_{r} \\ q_{g} \\ q_{b} \end{matrix} \right)
\rightarrow q^{'} = exp \left( ig_{s} \sum_{k=1}^{8} \eta_{k}(x)
\frac{\lambda_k}{2} \right) q
\end{equation}

Here the $\lambda_{k}$ are the Gell-Mann generators for $SU(3)$, $\eta(x)_{k}$ is the
space-time dependency for each generator, and  $g_s$ is the strong coupling constant.
As with the electroweak Lagrangian, the introduction of these space-time dependent terms adds new
terms into the kinematic portion of the Lagrangian and spoils the gauge 
invariance.  Again, a covariant derivative is introduced 

\begin{equation}
D_{\mu} = \partial_{\mu} - ig_{s}G_{\mu}^{k}\frac{\lambda_{k}}{2}
\end{equation}

to restore gauge invariance. The $G_{\mu}^{k}$ are the new fields introduced
for the 8 gluons.  These new fields transform under $SU(3)$ as

\begin{equation} \label{eq:qcd:gluon_field}
G_{\mu}^{k} \rightarrow G_{\mu}^{'k} = G_{\mu}^{k} + \partial_{\mu}\eta_{k}(x) +
g_{s}f_{klm}\eta_{l}(x)G_{\mu}^{m}
\end{equation}

Given these definitions the QCD Lagrangian ($\mathcal{L}_{QCD}$) can be
constructed as 

\begin{equation} \label{eq:qcd:qcd_lagrangian}
\mathcal{L}_{QCD} = \bar{q}(i\gamma_{\mu}D^{\mu} - m_{q})q -
\frac{1}{4}G_{k}^{\mu\nu}G_{k\mu\nu}
\end{equation}

where the gluon field tensor $G_{k}^{\mu\nu}$ is defined as

\begin{equation} \label{eq:qcd:gluon_tensor}
G_{k}^{\mu\nu} = \partial^{\mu}G_{k}^{\nu} - \partial^{\nu}G_{k}^{\mu} +
g_{s}f_{klm}G_{\l}^{\mu}G_{m}^{\nu}.
\end{equation}

The strong force is peculiar in that experiments observe only colorless objects
in the form of bound states of quarks known as hadrons.  Qualitatively, the
potential between quarks in a bound state (meson or baryon) gets stronger with
separation, unlike the other forces.  At the point where the system would
separate into color-charged objects, it becomes energetically favorable to
produce a quark/anti-quark pair in a process known as hadronization.  In other
words, attempting to separate a bound quark state into its colored constituents
simply results in new colorless bound states.  This requirement of colorless
objects by the strong force is known as color confinement. For highly energetic
strong interactions at hadron colliders the result is an expanding chain of
hadronizing quarks and gluons and their decay products known as a jet.


\section{Spontaneous Symmetry Breaking} \label{sec:theory:ssb}

Spontaneous symmetry breaking occurs when a system loses an inherent symmetry in
order to attain a lower energy configuration.

\section{The Higgs Mechanism} \label{sec:theory:higgs}

The Higgs mechanism is the system by which the gauge bosons and fermions gain
mass through the spontaneous breaking of the electroweak symmetry of the Higgs
potential \cite{Higgs:1964ia,Higgs:1966ev,Thomson:2013zua}.  This section will
also discuss briefly the couplings of the Higgs boson to massive particles, as
well as its self couplings.

\subsection{Electroweak Symmetry Breaking}

The Higgs field is expressed as a complex doublet, $\boldsymbol{\Phi}$, and thus
has four components defined as

\begin{equation} \label{eq:higgs:higgs_field}
\boldsymbol{\Phi}(x) = \left( \begin{matrix} \phi^{+} \\ \phi^{0} \end{matrix}
\right) = \frac{1}{\sqrt{2}} \left( \begin{matrix} \phi_{1}(x) + i\phi_{2}(x) \\
\phi_{3}(x) + i\phi_{4}(x) \end{matrix} \right).
\end{equation}

The four components of this field each represent a degree of freedom which
become the longitudinal polarizations of the $W^{\pm},Z$ gauge bosons and the
Higgs boson.  The resulting Lagrangian for the Higgs includes a kinetic term
(K) as well as the Higgs potential (V), all of which are invariant under the
electroweak gauge symmetry $SU(2)_L \times U(1)_Y$.  The definition is

\begin{equation} \label{eq:higgs:lagrangian}
\mathcal{L}_{\text{Higgs}} =
\underbrace{(D_{\mu}\boldsymbol{\Phi)^{\dagger}}D^{\mu}\boldsymbol{\Phi}}_{\text{K}}
- (\underbrace{\mu^{2}\boldsymbol{\Phi}^{\dagger}\boldsymbol{\Phi} +
  \lambda(\boldsymbol{\Phi}^{\dagger}\boldsymbol{\Phi})^{2}}_{\text{V}}).
\end{equation}

Here the $\mu^{2} < 0$ and $\lambda > 0$ are constrained such that the
potential forms a ring stable minima.  The shape of this potential is shown in
\Cref{fig:higgs_potential} and is often described as the ``Mexican-hat" or
"wine-bottle" potential. 

\begin{figure}[!h]
  \begin{center}
    \includegraphics[width=0.4\linewidth]{figures/theory/higgs_potential.png}
    \caption{ A lower-dimensionality representation of the shape of the Higgs
potential.  The central peak represents a rotationally symmetric
unstable state, while the trough represents the infinite number of minima that
can be selected upon the spontaneous breaking of the symmetry.}
    \label{fig:higgs_potential}
  \end{center}
\end{figure}

The value of this minima can be calculated by taking the derivative of V with
respect to $\boldsymbol{\Phi}$ and setting it equal to $0$. This value, also
known as the vacuum expectation value (vev) has been found to be $v \equiv
\sqrt{-\mu^{2}/\lambda} = 246$ GeV. The arbitrary vev of the ground state Higgs
field is acquired when the symmetry of the Higgs potential is spontaneously broken.
For ease of calculation the coordinate system is oriented such that

\begin{equation}
\left\langle \boldsymbol{\Phi}(x) \right\rangle = \frac{1}{\sqrt{2}} \left(
\begin{matrix} 0 \\ v \end{matrix} \right)
\end{equation} 

Next small perturbations around the minimum of the Higgs
potential are parameterized as 

\begin{equation} \label{eq:higgs:broken_higgs}
\left\langle \boldsymbol{\Phi}(x) \right\rangle = \frac{1}{\sqrt{2}} \left(
\begin{matrix} 0 \\ v + h(x) \end{matrix} \right) \text{exp} \left(
i\frac{\tau^{i}}{2}\theta^{i}(x) \right)
\end{equation} 

Here the real scalar field $h(x)$ corresponds to radial perturbations of the
minima, and the three $\theta^{i}(x)$ are the Nambu-Goldstone fields with
values determined by the choice of gauge.  Choosing the unitary gauge of
$\theta^{i}(x) = 0$ and expanding the kinetic term of
\Cref{eq:higgs:lagrangian} around the vev gives

\begin{equation} \label{eq:higgs:boson_masses}
\mathcal{L}_{\text{Higgs},K} = \frac{g^{2}v^{2}}{8} \left(
(W_{\mu}^{-})^{\dagger}W^{-\mu} + (W_{\mu}^{+})^{\dagger}W^{+\mu} \right) +
\frac{1}{2} \left( \begin{matrix} W_{\mu}^{3\dagger} & B_{\mu}^{\dagger}
\end{matrix} \right) \boldsymbol{M}^{2} \left( \begin{matrix} W^{3\mu} \\ B^{\mu}
\end{matrix} \right) + \ldots 
\end{equation}

Here the first term is the physical mass term for the $W^{\pm}$ bosons where
these charge eigenstates have been constructed out of the $W^{1,2}$ fields as
such $W^{\pm} = \frac{1}{\sqrt{2}}(W^{1} \mp iW^{2})$.  The second term
represents the mixture of the $W^{3}$ and $B$ fields through the mass matrix
$\boldsymbol{M}$.  Diagonalizing this matrix ($M_{D}$) and identifying the mass
eigenstates gives the physical fields of the photon ($\gamma$) and the $Z$
boson


\begin{equation}
\boldsymbol{M}_{D}^{2} = \left( \begin{matrix} 0 & 0 \\ 0 &
\frac{v^{2}}{4}(g_{W}^{2} + g^{'2)}   \end{matrix} \right)
\end{equation}

The upper left diagonal element corresponds to the massless photon
while the lower right diagonal element gives the mass of the massive $Z$ boson.
This leaves us with the following masses for the 4 electroweak bosons

\begin{equation}
m_{W} = \frac{1}{2}g_{W}v \quad , \quad m_Z = \frac{1}{2}v\sqrt{g_{W}^{2} + g^{'2}}
\quad , \quad m_\gamma = 0
\end{equation}

The masses of the $W^{\pm}$ and $Z$ gauge bosons can be related through the
Weinberg mixing angle defined as

\begin{equation}
\theta_W = \cos^{-1}\left( \frac{g_{W}}{\sqrt{g_{W}^{2}+g^{'2}}} \right) \rightarrow m_{Z} =
\frac{m_{W}}{\cos{\theta_{W}}}
\end{equation}

Using this definition one can write out the exact mixture of $B$ and $W^{3}$ that
make up the photon and $Z$ boson as

\begin{align}
\gamma &= \text{cos}(\theta_{W})B + \text{sin}(\theta_{W})W^{3} \\
Z &= -\text{sin}(\theta_{W})B + \text{cos}(\theta_{W})W^{3}
\end{align}

\subsection{Fermion Mass Terms} \label{sec:theory:fermion_mass}

\Cref{sec:theory:gsw} shows how a simple fermion mass terms violate gauge
invariance due to the mixing of the left and right chiral states.  The Higgs
mechanism, however, allows for a gauge-invariant method of generating mass
terms through the Yukawa coupling of the Higgs field to the fermion fields.  An
example is the Yukawa coupling term for a quark doublet
($\boldsymbol{\Psi}_{L}$) and singlet ($\Psi_R$) coupling to the Higgs field
($\boldsymbol{\Phi}$) after spontaneous symmetry breaking, with the form
shown in \Cref{eq:higgs:broken_higgs}, when the unitary gauge $\Phi^{i}(x) = 0$
is chosen.

\begin{align}
\mathcal{L}_{\text{Yukawa}} &= - g_{b} \left[ \boldsymbol{\bar{\Psi}_L}
\boldsymbol{\Phi} \Psi_R + \bar{\Psi}_{R} \boldsymbol{\Phi}^{\dagger} \boldsymbol{\Psi_L}
\right] \\ &= - \frac{g_{b}}{\sqrt{2}} \left[ \left( \begin{matrix}
\bar{t} & \bar{b} \end{matrix} \right)_L \left( \begin{matrix} 0 \\ v +
h \end{matrix} \right) b_{R} + \bar{b}_{R} \left( \begin{matrix} 0 & (v + h)
\end{matrix} \right) \left( \begin{matrix} t \\ b \end{matrix} \right)_L \ \right] \\ &= - \underbrace{\frac{g_{b}}{\sqrt{2}}
v}_{m_{b}} \left( \bar{b}_{L}b_{R} + \bar{b}_{R}b_{L}  \right)
- \underbrace{\frac{g_{b}}{\sqrt{2}}}_{g_{b,h}} h \left(
\bar{b}_{L}b_{R} + \bar{b}_{R}b_{L}  \right) 
\end{align}

In this way mass terms are generated for the fermion field and the gauge
invariance of the Lagrangian is maintained by using only gauge invariant
fields.  This operation has also left us with the second term which represents
the coupling of the bottom quark to the higgs itself thus giving us the form of its
coupling constant $g_{b,h}$.  Using this newly found mass of the bottom quark
$m_{b}$ the coupling can be written as

\begin{equation}
g_{b,h} = \frac{g_{b}}{\sqrt{2}} = \frac{m_{b}}{v}.
\end{equation}

Thus the coupling of the Higgs boson to a fermion is proportional
to the mass of the fermion itself. 
 
\subsection{The Higgs Boson}

The Higgs mechanism not only properly mixes the gauge fields to provide the
correct gauge-invariant mass terms, but also properly combines the left and
right chiral states of fermions to produce their mass terms. Furthermore, the 4
degrees of freedom in the Higgs doublet (\Cref{eq:higgs:higgs_field} become
physical particles, known as the Goldstone bosons, during the spontaneous
breaking of the electroweak potential \cite{Higgs:1964pj}.  Three of these
Goldstone bosons are ``eaten" by the 3 gauge bosons $W^{\pm}$ and $Z$ to form
their longitudinal components thus give them their mass. The final broken
degree of freedom is absorbed by the new massive scalar particle, the Higgs
boson \cite{Higgs:1964pj}.

Focusing on the potential term (V) of
\Cref{eq:higgs:lagrangian} and substituting in the definition for
$\boldsymbol{\Phi}$ given in \Cref{eq:higgs:broken_higgs} gives

\begin{equation}
\mathcal{L}_\text{Higgs,V} = \frac{1}{2} \mu^{2} v^{2} - \mu^{2} h^{2} +
\lambda v h^{3} + \frac{1}{4} \lambda h^{4}
\end{equation}

The first term is constant and thus can be ignored.  The second term is the
mass term for the Higgs boson, $m_h = \sqrt{-2\mu^{2}} = \sqrt{2\lambda}v$.
Remembering that $h = h(x)$ was used for small radial perturbations of the
Higgs field the Higgs boson can be identified as a radial excitation of the
Higgs field.  Finally, the third and fourth terms represent the Higgs boson
self-couplings.  With these couplings and mass terms in hand the next step is
to experimentally verify this theory as discussed next in \Cref{chap:higgs}.

\section{Parton Distribution Function} \label{sec:theory:pdf}

Before QFT the proton was thought to be a hard ball containing no smaller
constituents.  However, we know now that that the strong field inside the proton
allows for any strong object to exist with some probability which changes based
off of the total energy of the proton.  This behavior is represented then by a
Probability Distribution Function.


