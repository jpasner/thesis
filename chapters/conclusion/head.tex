\chapter{Conclusion} \label{chap:conclusion}

A search for boosted $H \rightarrow b\bar{b}$ was performed using an integrated
luminosity of $80.5 \text{fb}^{-1}$ of LHC proton-proton collision data
recorded at the ATLAS experiment with a center-of-mass energy of $\sqrt{s} =
13~\TeV$. The analysis of this data measured a signal strength of $\mu_{H} =
5.8 \pm 3.1~\mathrm{(stat.)} \pm 1.9~\mathrm{(syst.)} \pm
1.7~\mathrm{(theo.)}$, which is 1.6 standard deviations higher than the
background-only hypothesis with an expected sensitivity of $0.28\sigma$.  The
CMS collaboration performed a similar analysis in 2017 with an integrated
luminosity of $39.5~\ifb$ and found a signal strength for boosted $H
\rightarrow b\bar{b}$ of $\mu_{H} = 2.3 \pm
1.5~\text{(stat.)}_{-0.4}^{+1.0}~\text{(syst.)}$ which is consistent with this
analysis' measurement within 2 standard deviations \cite{Sirunyan:2017dgc}.

This is the first time this search has been performed in ATLAS and thus it
represents important advancements in the use of boosted jet techniques and the
implementation of new techniques such as the variable radius jets.  Future work
in this novel channel will be necessary to optimize these new techniques and to
reduce the major systematic uncertainties described in \Cref{chap:systematics}.  The ATLAS
calorimeter design gives an excellent energy resolution,
making software improvements to jet technologies an important avenue of
research for future versions of this analysis. Within ATLAS there is active
work to include a  particle flow algorithm into the large-$R$ jet
reconstruction. This algorithm combines calorimeter measurements of
neutral-particle energies with information from charged hadron tracks in the ID
to give better jet resolution.  Furthermore, studies of new substructure-based
triggers offer future improvements to the signal event selection.

In conclusion, this dissertation provides a measurement in the Higgs sector
representing a new contribution to the field, serves as a validation of the
predictions of the Standard Model, and presents new techniques for the analysis
of boosted signatures in the ATLAS detector.
