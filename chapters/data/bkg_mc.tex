\section{Background Monte Carlo} \label{sec:data:bkg_mc}

Modeling the expected contributions of backgrounds to the analysis is done with
a mix of data driven methods and Monte Carlo simulated samples as discussed in
\Cref{chap:background}.  The MC samples were used for the development of the
modeling of the non-resonant QCD multijet backgrounds and for the estimation
of the major resonant backgrounds from $V$+jet, $t\bar{t}$, and single-top
production.  For the multijet background the final estimation is data driven,
but the MC was used to develop the background model.

The QCD dijet events were simulated by the \textsc{Pythia} 8.186
generator~\cite{Sjostrand:2007gs} with the A14 tune and the NNPDF23 LO
PDF~\cite{Carrazza:2013axa} using \textsc{EvtGen} to decay the resulting
$b$-hadrons~\cite{LANGE2001152}.  To maintain a constant statistical precision
over a large momentum range, the weighted events were generated with a flat \pT
spectrum.

Hadronically decaying $W$ and $Z$ events were produced with a maximum of four
additional partons at leading order (LO).  This was accomplished with the
\textsc{Sherpa} 2.1.1 generator~\cite{Gleisberg:2008ta} and the CT10 parton
distribution functions~\cite{Lai:2010vv}.  For leptonically decaying $W$ and
$Z$ events samples were produced with a maximum of two additional partons at LO
and a maximum of two at next-to-leading order (NLO).  Next the LO hadronic $W$
and $Z$ cross sections were corrected by applying multiplicative ``$k$-factors"
derived from the corresponding NLO leptonic $W$ and $Z$ samples.  These corrections
were 1.28 for the $W$+jets and 1.37 for the $Z$+jets \cite{Aaboud:2018zba}. An
alternate sample of hadronically decaying $W$ and $Z$ events was produced for
cross checks using the \text{Herwig}++2.7.1 generator~\cite{Bahr:2008pv} with
the UEEE4 tune~\cite{Buckley:2018wdv} and the CTEQ6L1~\cite{Pumplin:2002vw}.
Unlike the \textsc{Sherpa} samples these \textsc{Herwig} samples only contained
one additional parton in the matrix element.

Our $t\bar{t}$ samples were generated using \textsc{Powheg-Box}
2~\cite{Campbell2012} and the NNPDF30 NLO parton distribution
functions~\cite{Ball:2014uwa}. After generation the events were showered using
\textsc{Pythia} 8.230~\cite{Sjostrand:2014zea} using the A14 tune and the
NNPDF23 LO parton distribution functions~\cite{Carrazza:2013axa} with all
$b$-hadron decays performed by \textsc{EvtGen}~\cite{LANGE2001152}.  The
samples were then broken up according to the decay mode of the two top quarks
into three categories; all-hadronic, semi-leptonic, all-leptonic. Additional
$t\bar{t}$ events were generated with the \textsc{Sherpa} 2.2.1
generator~\cite{Gleisberg:2008ta} using the NNPDF30 NNLO parton distribution
functions~\cite{Ball:2014uwa}.  This second sample was used as a cross check of
the main samples generated with \textsc{Powheg-Box} 2 + \textsc{Pythia} 8.

Single-top samples, containing a single top/anti-top quark and a
$W^{-}$/$W^{+}$, were generated with \textsc{Powheg-Box}
2~\cite{Campbell2012} with the NNPDF30 parton distribution
functions~\cite{Lai:2010vv}. This process was showered using \textsc{Pythia}
8.230~\cite{Sjostrand:2014zea} configured with the A14 tune and the NNPDF23
parton distribution functions~\cite{Carrazza:2013axa} with all resulting
$b$-hadrons decayed via \textsc{EvtGen}~\cite{LANGE2001152}.
