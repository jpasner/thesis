\chapter{Data and Monte-Carlo Simulation} \label{chap:data}

This analysis focuses on the data collected by the ATLAS detector from \pp
collisions produced by the LHC at the center-of-mass-energy of 13 TeV.  In
particular the analysis shown uses datasets collected in 2015, 2016, and 2017
and ammounts to an integrated luminosity of $80.5 \text{ fb}^{-1}$ after beam,
detector and data-quality requirements are taken into account. 

In order to compare our findings with theory, we use the predictions of the SM
to produce Monte-Carlo (MC) simulated events to model the signal and background
processes.  These MC samples go through a full simulation of the ATLAS detector
and are reconstructed using the same algorithms as used on data such that the
MC and Data have the same format at analysis level. This allows us to analyze
the MC and Data using the same framework such that we can make direct
comparisons between theory and reality as our final product.

The following sections discuss the systems for selecting the data used in the
analysis as well as the software packages, developed in collaboration with
theorists, used to simulate the signal and background processes of the
analysis.

\section{Data Used} \label{sec:data:data}

As mentioned earlier, the data are checked to make sure they are of high
quality, meaning that the beam, detector and data collection systems were all
fully operational during the event in question. These data quality requirements
are enforced by choosing only events from each respective year's Good Runs List
(GRL), an XML file produced by the ATLAS data quality monitoring team that
lists all luminosity blocks that have met the data quality criteria.  This
analysis uses three such GRLs - one for each year of data taking
(2015, 2016, 2017) - corresponding to annual integrated luminosities of
$3.2~\ifb$, $33~\ifb$, and $44.3~\ifb$.

\section{Higgs Boson Signal Monte Carlo Samples} \label{sec:data:signal_mc}

In order to simulate Higgs boson events, the three leading production
mechanisms at the LHC were considered, shown in \Cref{fig:higgs_production}:
gluon-gluon fusion, vector boson fusion and Higgsstrahlung.  These three
production modes represent 50\%, 30\% and 20\% of the total Higgs signal,
respectively, before analysis cuts are applied.  In all samples the simulated
Higgs boson is forced to decay to the $b\bar{b}$ final state.

The ggF $H$ + jet events were generated using the HJ+MiNLO~\cite{Hamilton2015}
prescription, assuming a finite top quark mass, with the \textsc{Powheg-Box} 2
generator~\cite{Campbell2012} and the NNPDF30 NNLO parton distribution
functions~\cite{Hamilton:2012rf}.  After generation the events were showered
using \textsc{Pythia} 8.212~\cite{Sjostrand:2014zea} with the AZNLO tune and
the CTEQ6L1 parton distribution functions~\cite{Pumplin:2002vw}. Any
$b$-hadrons produced during this process were decayed using
\textsc{EvtGen}~\cite{LANGE2001152}.

The VBF $H$ + jet events were also generated using the \textsc{Powheg-Box}
generator~\cite{Nason:2009ai} with the NNPDF30 NLO parton distribution
functions~\cite{Hamilton:2012rf}. Again the showering was performed with
\textsc{Pythia} 8.212~\cite{Sjostrand:2014zea} using the AZNLO tune and the
CTEQ6L1 parton distribution functions~\cite{Pumplin:2002vw}.  The decay of
$b$-hadrons was again performed using \textsc{EvtGen}~\cite{LANGE2001152}.

Higgsstrahlung events were generated using the \textsc{Pythia} 8.212
generator~\cite{Sjostrand:2014zea}, the AZNLO tune and the CTEQ6L1 parton
distribution functions~\cite{Pumplin:2002vw}.  Again the decay of $b$-hadrons
is handled by \textsc{EvtGen}~\cite{LANGE2001152}. Unfortunately the
\textsc{Pythia} $ZH$ process does not include the $gg \rightarrow ZH$
contribution.  To account for this the cross section is corrected to the LHC
Higgs cross section working group's (LHCHXSWG) recommendation
\cite{MelladoGarcia:2150771}.

\section{Background Monte Carlo} \label{sec:data:bkg_mc}

Modeling the expected contributions of backgrounds to the analysis is done with
a mix of data driven methods and Monte Carlo simulated samples as discussed in
\Cref{chap:background}.  The MC samples were used for the development of the
modeling of the non-resonant QCD multijet backgrounds and for the estimation
of the major resonant backgrounds from $V$+jet, $t\bar{t}$, and single-top
production.  For the multijet background the final estimation is data driven,
but the MC was used to develop the background model.

The QCD dijet events were simulated by the \textsc{Pythia} 8.186
generator~\cite{Sjostrand:2007gs} with the A14 tune and the NNPDF23 LO
PDF~\cite{Carrazza:2013axa} using \textsc{EvtGen} to decay the resulting
$b$-hadrons~\cite{LANGE2001152}.  To maintain a constant statistical precision
over a large momentum range, the weighted events were generated with a flat \pT
spectrum.

Hadronically decaying $W$ and $Z$ events were produced with a maximum of four
additional partons at leading order (LO).  This was accomplished with the
\textsc{Sherpa} 2.1.1 generator~\cite{Gleisberg:2008ta} and the CT10 parton
distribution functions~\cite{Lai:2010vv}.  For leptonically decaying $W$ and
$Z$ events samples were produced with a maximum of two additional partons at LO
and a maximum of four at next-to-leading order (NLO).  This allowed us to
correct the LO hadronic $W$ and $Z$ cross-sections to the NLO leptonic $W$ and
$Z$ cross-sections by applying multiplicative ``$k$-factors."  These corrections
were 1.28 for the $W$+jets and 1.37 for the $Z$+jets \cite{Aaboud:2018zba}. An
alternate sample of hadronically decaying $W$ and $Z$ events was produced for
cross checks using the \text{Herwig}++2.7.1 generator~\cite{Bahr:2008pv} with
the UEEE4 tune~\cite{Buckley:2018wdv} and the CTEQ6L1~\cite{Pumplin:2002vw}.
Unlike the \textsc{Sherpa} samples these \textsc{Herwig} samples only contained
one additional parton in the matrix element calculation.

Our $t\bar{t}$ samples were generated at tree-level using \textsc{Powheg-Box}
2~\cite{Campbell2012} and the NNPDF30 NLO parton distribution
functions~\cite{Ball:2014uwa}. After generation the events were showered using
\textsc{Pythia} 8.230~\cite{Sjostrand:2014zea} using the A14 tune and the
NNPDF23 LO parton distribution functions~\cite{Carrazza:2013axa} with all
$b$-hadron decays performed by \textsc{EvtGen}~\cite{LANGE2001152}.  The
samples were then broken up according to the decay mode of the two top quarks
into three categories; all-hadronic, semi-leptonic, all-leptonic. Additional
$t\bar{t}$ events were generated with the \textsc{Sherpa} 2.2.1
generator~\cite{Gleisberg:2008ta} using the NNPDF30 NNLO parton distribution
functions~\cite{Ball:2014uwa}.  This second sample was used as a cross check of
the main samples generated with \textsc{Powheg-Box} 2 + \textsc{Pythia} 8.

Single-top samples, containing a single top/anti-top quark and a
$W^{-}$/$W^{+}$, were generated at tree level with \textsc{Powheg-Box}
2~\cite{Campbell2012} with the NNPDF30 parton distribution
functions~\cite{Lai:2010vv}. This process was showered using \textsc{Pythia}
8.230~\cite{Sjostrand:2014zea} configured with the A14 tune and the NNPDF23
parton distribution functions~\cite{Carrazza:2013axa} with all resulting
$b$-hadrons decayed via \textsc{EvtGen}~\cite{LANGE2001152}.

