\chapter{Data and Monte-Carlo Simulation} \label{chap:data}

This analysis focuses on the data collected by the ATLAS detector from \pp
collisions produced by the LHC at the center-of-mass-energy of 13 TeV.  In
particular the analysis shown uses datasets collected in 2015, 2016, and 2017
and ammounts to an integrated luminosity of $80.5 \text{ fb}^{-1}$ after beam,
detector and data-quality requirements are taken into account. 

In order to compare our findings with theory, we use the predictions of the SM
to produce Monte-Carlo (MC) simulated events to model the signal and background
processes.  These MC samples go through a full simulation of the ATLAS detector
and are reconstructed using the same algorithms as used on data such that the
MC and Data have the same format at analysis level. This allows us to analyze
the MC and Data using the same framework such that we can make direct
comparisons between theory and reality as our final product.

The following sections discuss the systems for selecting the data used in the
analysis as well as the software packages, developed in collaboration with
theorists, used to simulate the signal and background processes of the
analysis.

\section{Data Used} \label{sec:data:data}

As mentioned before the data used is checked to make sure it is of high
quality, meaning that the beam, detector and data collection systems were all
fully opperational during the event in question. These data quality
requirements are enfored by choosing only events from each respective years
Good Runs List (GRL), an XML file produced by the ATLAS data quality
monitoring team that lists all events that have met the data quality critera.
This analysis uses three such GRLs - one for each year of data taking
(2015,2016,2017) - corresponding to annual integrated luminosities of $3.2
\text{ fb}^{-1}$, $33 \text{ fb}^{-1}$, and $44.3 \text{ fb}^{-1}$.

\section{Higgs Boson Signal Monte Carlo Samples} \label{sec:data:signal_mc}

In order to simulate Higgs boson events, the three leading production
mechanisms at the LHC were considered, shown in \Cref{fig:higgs_production}:
gluon-gluon fusion, vector boson fusion and Higgsstrahlung.  These three
production modes represent approximately 88\%, 7\% and 4\% of the total Higgs
signal, respectively, before analysis cuts are applied.  In all samples the
simulated Higgs boson is forced to decay to the $b\bar{b}$ final state.
 
The ggF $H$ + jet events were generated using the HJ+MiNLO~\cite{Hamilton2015}
prescription, assuming a finite top quark mass, with the \textsc{Powheg-Box} 2
generator~\cite{Campbell2012} and the NNPDF30 NNLO parton distribution
functions~\cite{Hamilton:2012rf}.  After generation the events were showered
using \textsc{Pythia} 8.212~\cite{Sjostrand:2014zea} with the AZNLO tune and
the CTEQ6L1 parton distribution functions~\cite{Pumplin:2002vw}. Any
$b$-hadrons produced during this process were decayed using
\textsc{EvtGen}~\cite{LANGE2001152}. During generation a parton level filter of
$k_{T} > 200~\GeV$ was used to select high momentum events.  This filtering is
reflected in the cross section in \Cref{table:data:signal}.

The VBF $H$ + jet events were also generated using the \textsc{Powheg-Box}
generator~\cite{Nason:2009ai} with the NNPDF30 NLO parton distribution
functions~\cite{Hamilton:2012rf}. Again the showering was performed with
\textsc{Pythia} 8.212~\cite{Sjostrand:2014zea} using the AZNLO tune and the
CTEQ6L1 parton distribution functions~\cite{Pumplin:2002vw}.  The decay of
$b$-hadrons was again performed using \textsc{EvtGen}~\cite{LANGE2001152}.
During generation the produced Higgs boson was required to have a $p_{T} >
250~\GeV$.  This filtering is reflected in the filtering efficiency in
\Cref{table:data:signal}.

Higgsstrahlung events were generated using the \textsc{Pythia} 8.212
generator~\cite{Sjostrand:2014zea}, the AZNLO tune and the CTEQ6L1 parton
distribution functions~\cite{Pumplin:2002vw}.  Again the decay of $b$-hadrons
is handled by \textsc{EvtGen}~\cite{LANGE2001152}. Unfortunately the
\textsc{Pythia} $ZH$ process does not include the $gg \rightarrow ZH$
contribution.  To account for this the cross section is corrected to the LHC
Higgs cross section working group's (LHCHXSWG) recommended
$\sigma_{\text{NLO+NLL}}^{\text{ggZH}}$ with NLO + Next to Leading Log (NLL)
accuracy \cite{MelladoGarcia:2150771}. During generation a filter of $p_{T} >
350~\GeV$ was used to select for high momentum events. This
filtering is reflected in the filtering efficiency in \Cref{table:data:signal}.

\begin{sidewaystable}
  \centerline{
  \centering
  {\tiny
  \begin{tabular}{l|c|cccc}
    \textbf{Dataset} & \textbf{Campaign} & \textbf{xsec (pb)} & \shortstack{\textbf{Filter} \\ \textbf{Efficiency}} & \shortstack{\textbf{Events} \\ \textbf{Generated}} & \shortstack{\textbf{Effective} \\  \textbf{Luminosity (\ifb)}} \\
    \hline
    \hline
    309450.PowhegPy8EG\_NNLOPS\_nnlo\_30\_ggH125\_bb\_kt200 & MC16a & 0.4763 & 1.0 & 400000 & 839.1 \\
    345931.PowhegPy8EG\_NNPDF30\_AZNLOCTEQ6L1\_VBFH125\_bb & MC16a & 3.7296 & 0.026877 & 499500 & 4794 \\
    309451.Pythia8EvtGen\_A14NNPDF23LO\_WH125\_bb\_fj350 & MC16a & 0.63487 & 0.01306 & 100000 & 12061 \\
    309452.Pythia8EvtGen\_A14NNPDF23LO\_ZH125\_bb\_fj350 & MC16a & 0.34656 & 0.013036 & 100000 & 22135 \\
    \hline
    309450.PowhegPy8EG\_NNLOPS\_nnlo\_30\_ggH125\_bb\_kt200 & MC16d & 0.4763 & 1.0 & 499000 & 1047 \\
    345931.PowhegPy8EG\_NNPDF30\_AZNLOCTEQ6L1\_VBFH125\_bb & MC16d & 3.7296 & 0.026877 & 650000 & 6238 \\
    309451.Pythia8EvtGen\_A14NNPDF23LO\_WH125\_bb\_fj350 & MC16d & 0.63487 & 0.01306 & 130000 & 15679 \\
    309452.Pythia8EvtGen\_A14NNPDF23LO\_ZH125\_bb\_fj350 & MC16d & 0.34656 & 0.013036 & 130000 & 28775 \\
  \end{tabular}}
  }
\caption{List of datasets used for Higgs boson Monte Carlo \cite{Krizka:2310645}.}
\label{table:data:signal}
\end{sidewaystable}



\section{Background Monte Carlo} \label{sec:data:bkg_mc}

Modeling the expected contributions of backgrounds to the analysis is done with
a mix of data driven methods and Monte Carlo simulated samples as discussed in
\Cref{chap:background}.  The MC samples were used for the development of the
modeling of the non-resonant QCD multijet backgrounds and for the estimation
of the major resonant backgrounds from $V$+jet, $t\bar{t}$, and single-top
production.  For the multijet background the final estimation is data driven,
but the MC was used to develop the background model.

The QCD dijet events were simulated by the \textsc{Pythia} 8.186
generator~\cite{Sjostrand:2007gs} with the A14 tune and the NNPDF23 LO
PDF~\cite{Carrazza:2013axa} using \textsc{EvtGen} to decay the resulting
$b$-hadrons~\cite{LANGE2001152}.  To maintain a constant statistical precision
over a large momentum range, the weighted events were generated with a flat \pT
spectrum. The QCD samples used to develop the background model are summarized
in \Cref{table:data:QCD}.

\begin{sidewaystable}
  \centerline{
  \centering
  {\tiny
  \begin{tabular}{l|c|cccc}
    \textbf{Dataset} & \textbf{Campaign} & \textbf{xsec (pb)} & \shortstack{\textbf{Filter} \\ \textbf{Efficiency}} & \shortstack{\textbf{Events} \\ \textbf{Generated}} & \shortstack{\textbf{Effective} \\  \textbf{Luminosity (\ifb)}} \\
    \hline
    \hline
    361020.Pythia8EvtGen\_A14NNPDF23LO\_jetjet\_JZ0W & MC16a & 78420e+06 & 0.9755 & 16000000 & 2.0915e-07 \\
    361021.Pythia8EvtGen\_A14NNPDF23LO\_jetjet\_JZ1W & MC16a & 78420e+06 & 0.00067143 & 15998000 & 7.5762e-05 \\
    361022.Pythia8EvtGen\_A14NNPDF23LO\_jetjet\_JZ2W & MC16a & 24332e+05 & 0.00033423 & 15989500 & 0.0053309 \\
    361023.Pythia8EvtGen\_A14NNPDF23LO\_jetjet\_JZ3W & MC16a & 26454e+03 & 0.00032016 & 15879500 & 0.44989 \\
    361024.Pythia8EvtGen\_A14NNPDF23LO\_jetjet\_JZ4W & MC16a & 25463e+01 & 0.00053138 & 15925500 & 29.858 \\
    361025.Pythia8EvtGen\_A14NNPDF23LO\_jetjet\_JZ5W & MC16a & 4553.5 & 0.00092409 & 15993500 & 1103.2 \\
    361026.Pythia8EvtGen\_A14NNPDF23LO\_jetjet\_JZ6W & MC16a & 257.53 & 0.00094242 & 17834000 & 25916 \\
    361027.Pythia8EvtGen\_A14NNPDF23LO\_jetjet\_JZ7W & MC16a & 16.215 & 0.0003928 & 15983000 & 35888 \\
    361028.Pythia8EvtGen\_A14NNPDF23LO\_jetjet\_JZ8W & MC16a & 0.62503 & 0.010176 & 15999000 & 25154e+02 \\
    361029.Pythia8EvtGen\_A14NNPDF23LO\_jetjet\_JZ9W & MC16a & 0.019639 & 0.012076 & 13995500 & 59013e+03 \\
    361030.Pythia8EvtGen\_A14NNPDF23LO\_jetjet\_JZ10W & MC16a & 0.0011962 & 0.0059087 & 13985000 & 19786e+05 \\
    361031.Pythia8EvtGen\_A14NNPDF23LO\_jetjet\_JZ11W & MC16a & 4.2259e-05 & 0.0026761 & 15948000 & 1.4102e+11 \\
    361032.Pythia8EvtGen\_A14NNPDF23LO\_jetjet\_JZ12W & MC16a & 1.0367e-06 & 0.00042592 & 15995600 & 3.6226e+13 \\
    \hline
    361020.Pythia8EvtGen\_A14NNPDF23LO\_jetjet\_JZ0W & MC16d & 78420e+06 & 0.9755 & 15987000 & 2.0898e-07 \\
    361021.Pythia8EvtGen\_A14NNPDF23LO\_jetjet\_JZ1W & MC16d & 78420e+06 & 0.00067143 & 15997000 & 7.5728e-05 \\
    361022.Pythia8EvtGen\_A14NNPDF23LO\_jetjet\_JZ2W & MC16d & 24332e+05 & 0.00033423 & 15981000 & 0.005331 \\
    361023.Pythia8EvtGen\_A14NNPDF23LO\_jetjet\_JZ3W & MC16d & 26454e+03 & 0.00032016 & 15878500 & 0.45043 \\
    361024.Pythia8EvtGen\_A14NNPDF23LO\_jetjet\_JZ4W & MC16d & 25463e+01 & 0.00053138 & 15974500 & 29.968 \\
    361025.Pythia8EvtGen\_A14NNPDF23LO\_jetjet\_JZ5W & MC16d & 4553.5 & 0.00092409 & 15991500 & 1103.6 \\
    361026.Pythia8EvtGen\_A14NNPDF23LO\_jetjet\_JZ6W & MC16d & 257.53 & 0.00094242 & 17880400 & 25979 \\
    361027.Pythia8EvtGen\_A14NNPDF23LO\_jetjet\_JZ7W & MC16d & 16.215 & 0.0003928 & 15116500 & 33886e+01 \\
    361028.Pythia8EvtGen\_A14NNPDF23LO\_jetjet\_JZ8W & MC16d & 0.62503 & 0.010176 & 15987000 & 25136e+02 \\
    361029.Pythia8EvtGen\_A14NNPDF23LO\_jetjet\_JZ9W & MC16d & 0.019639 & 0.012076 & 14511500 & 61189+03 \\
    361030.Pythia8EvtGen\_A14NNPDF23LO\_jetjet\_JZ10W & MC16d & 0.0011962 & 0.0059087 & 15988000 & 22620e+05 \\
    361031.Pythia8EvtGen\_A14NNPDF23LO\_jetjet\_JZ11W & MC16d & 4.2259e-05 & 0.0026761 & 15993000 & 1.4142e+11 \\
    361032.Pythia8EvtGen\_A14NNPDF23LO\_jetjet\_JZ12W & MC16d & 1.0367e-06 & 0.00042592 & 15640000 & 3.5421e+13 \\
  \end{tabular}}
  }
\caption{List of datasets used for QCD Monte Carlo \cite{Krizka:2310645}.}
\label{table:data:QCD}
\end{sidewaystable}

Hadronically decaying $W$ and $Z$ events were produced with a maximum of four
additional partons at leading order (LO).  This was accomplished with the
\textsc{Sherpa} 2.1.1 generator~\cite{Gleisberg:2008ta} and the CT10 parton
distribution functions~\cite{Lai:2010vv}.  For leptonically decaying $W$ and
$Z$ events samples were produced with a maximum of two additional partons at LO
and a maximum of two at next-to-leading order (NLO).  Next the LO hadronic $W$
and $Z$ cross sections were corrected by applying multiplicative ``$k$-factors"
derived from the corresponding NLO leptonic $W$ and $Z$ samples.  These
corrections were 1.28 for the $W$+jets and 1.37 for the $Z$+jets
\cite{Aaboud:2018zba}. An alternate sample of hadronically decaying $W$ and $Z$
events was produced for cross checks using the \text{Herwig}++2.7.1
generator~\cite{Bahr:2008pv} with the UEEE4 tune~\cite{Buckley:2018wdv} and the
CTEQ6L1~\cite{Pumplin:2002vw}.  Unlike the \textsc{Sherpa} samples these
\textsc{Herwig} samples only contained one additional parton in the matrix
element. The hadronic $W$+jets and $Z$+jets samples used in this
analysis~\footnote{Note that the leptonically decaying $W$ and $Z$
samples were only used to derive correction factors and thus are not
presented.} are summarized in \Cref{table:data:V_hadronic}.

\begin{sidewaystable}
  \centerline{
  \centering
  {\tiny
  \begin{tabular}{l|c|cccc}
    \textbf{Dataset} & \textbf{Campaign} & \textbf{xsec (pb)} & \shortstack{\textbf{Filter} \\ \textbf{Efficiency}} & \shortstack{\textbf{Events} \\ \textbf{Generated}} & \shortstack{\textbf{Effective} \\  \textbf{Luminosity (\ifb)}} \\
    \hline
    \hline
    304307.Sherpa\_CT10\_Wqq\_Pt280\_500 & MC16a & 29.487 & 1.0 & 150000 & 4.8429 \\
    304308.Sherpa\_CT10\_Wqq\_Pt500\_1000 & MC16a & 2.162 & 1.0 & 30000 & 12.646 \\
    304309.Sherpa\_CT10\_Wqq\_Pt1000 & MC16a & 0.046218 & 1.0 & 15000 & 273.61 \\
    \hline
    304307.Sherpa\_CT10\_Wqq\_Pt280\_500 & MC16d & 29.481 & 1.0 & 140000 & 4.5533 \\
    304308.Sherpa\_CT10\_Wqq\_Pt500\_1000 & MC16d & 2.1642 & 1.0 & 30000 & 13.431 \\
    304309.Sherpa\_CT10\_Wqq\_Pt1000 & MC16d & 0.046152 & 1.0 & 15000 & 287.55 \\
    \hline
    \hline
    304707.Sherpa\_CT10\_Zqq\_Pt280\_500 & MC16a & 12.623 & 1.0 & 75000 & 5.8079 \\
    304708.Sherpa\_CT10\_Zqq\_Pt500\_1000 & MC16a & 0.90771 & 1.0 & 15000 & 16.090 \\
    304709.Sherpa\_CT10\_Zqq\_Pt1000 & MC16a & 0.018344 & 1.0 & 10000 & 251.86 \\
    \hline
    304707.Sherpa\_CT10\_Zqq\_Pt280\_500 & MC16d & 12.644 & 1.0 & 75000 & 5.6273 \\
    304708.Sherpa\_CT10\_Zqq\_Pt500\_1000 & MC16d & 0.90886 & 1.0 & 15000 & 16.278 \\
    304709.Sherpa\_CT10\_Zqq\_Pt1000 & MC16d & 0.018264 & 1.0 & 10000 & 524.45 \\
    \hline
    \hline
    304673.Herwigpp\_UEEE5CTEQ6L1\_Wjhadronic\_280\_500 & MC16a & 13.491 & 1.0 & 150000 & 11.119 \\
    304674.Herwigpp\_UEEE5CTEQ6L1\_Wjhadronic\_500\_700 & MC16a & 0.91592 & 1.0 & 39000 & 42.580 \\
    304675.Herwigpp\_UEEE5CTEQ6L1\_Wjhadronic\_700\_1000 & MC16a & 0.17413 & 1.0 & 30000 & 172.29 \\
    304676.Herwigpp\_UEEE5CTEQ6L1\_Wjhadronic\_1000\_1400 & MC16a & 0.021936 & 1.0 & 30000 & 1367.6 \\
    304677.Herwigpp\_UEEE5CTEQ6L1\_Wjhadronic\_1400 & MC16a & 0.0022502 & 1.0 & 20000 & 8888.1 \\
    \hline
    304673.Herwigpp\_UEEE5CTEQ6L1\_Wjhadronic\_280\_500 & MC16d & 13.491 & 1.0 & 190000 & 14.083 \\
    304674.Herwigpp\_UEEE5CTEQ6L1\_Wjhadronic\_500\_700 & MC16d & 0.91592 & 1.0 & 50000 & 54.590 \\
    304675.Herwigpp\_UEEE5CTEQ6L1\_Wjhadronic\_700\_1000 & MC16d & 0.17413 & 1.0 & 40000 & 229.71 \\
    304676.Herwigpp\_UEEE5CTEQ6L1\_Wjhadronic\_1000\_1400 & MC16d & 0.021936 & 1.0 & 40000 & 1823.5 \\
    304677.Herwigpp\_UEEE5CTEQ6L1\_Wjhadronic\_1400 & MC16d & 0.0022502 & 1.0 & 30000 & 13332 \\
    \hline
    \hline
    304678.Herwigpp\_UEEE5CTEQ6L1\_Zjhadronic\_280\_500 & MC16a & 5.4672 & 1.0 & 75000 & 13.718 \\
    304679.Herwigpp\_UEEE5CTEQ6L1\_Zjhadronic\_500\_700 & MC16a & 0.37032 & 1.0 & 20000 & 54.007 \\
    304680.Herwigpp\_UEEE5CTEQ6L1\_Zjhadronic\_700\_1000 & MC16a & 0.070961 & 1.0 & 15000 & 211.38 \\
    304681.Herwigpp\_UEEE5CTEQ6L1\_Zjhadronic\_1000\_1400 & MC16a & 0.0087931 & 1.0 & 15000 & 1705.9 \\
    304682.Herwigpp\_UEEE5CTEQ6L1\_Zjhadronic\_1400 & MC16a & 0.00089549 & 1.0 & 10000 & 11167 \\
    \hline
    304678.Herwigpp\_UEEE5CTEQ6L1\_Zjhadronic\_280\_500 & MC16d & 5.4672 & 1.0 & 100000 & 18.291 \\
    304679.Herwigpp\_UEEE5CTEQ6L1\_Zjhadronic\_500\_700 & MC16d & 0.37032 & 1.0 & 30000 & 81.011 \\
    304680.Herwigpp\_UEEE5CTEQ6L1\_Zjhadronic\_700\_1000 & MC16d & 0.070961 & 1.0 & 20000 & 281.84 \\
    304681.Herwigpp\_UEEE5CTEQ6L1\_Zjhadronic\_1000\_1400 & MC16d & 0.0087931 & 1.0 & 20000 & 2274.5 \\
    304682.Herwigpp\_UEEE5CTEQ6L1\_Zjhadronic\_1400 & MC16d & 0.00089549 & 1.0 & 20000 & 22334 \\
  \end{tabular}}
  }
\caption{List of datasets used for hadronically decaying $W$+jets and $Z$+jets Monte Carlo \cite{Krizka:2310645}.}
\label{table:data:V_hadronic}
\end{sidewaystable}

Our $t\bar{t}$ samples were generated using \textsc{Powheg-Box}
2~\cite{Campbell2012} and the NNPDF30 NLO parton distribution
functions~\cite{Ball:2014uwa}. After generation the events were showered using
\textsc{Pythia} 8.230~\cite{Sjostrand:2014zea} using the A14 tune and the
NNPDF23 LO parton distribution functions~\cite{Carrazza:2013axa} with all
$b$-hadron decays performed by \textsc{EvtGen}~\cite{LANGE2001152}.  The
samples were then broken up according to the decay mode of the two top quarks
into three categories; all-hadronic, semi-leptonic, all-leptonic. Additional
$t\bar{t}$ events were generated with the \textsc{Sherpa} 2.2.1
generator~\cite{Gleisberg:2008ta} using the NNPDF30 NNLO parton distribution
functions~\cite{Ball:2014uwa}.  This second sample was used as a cross check of
the main samples generated with \textsc{Powheg-Box} 2 + \textsc{Pythia} 8. The
$t\bar{t}$ samples used in this analysis are summarized in
\Cref{table:data:ttbar}.

\begin{sidewaystable}
  \centerline{
  \centering
  {\tiny
  \begin{tabular}{l|c|cccc}
    \textbf{Dataset} & \textbf{Campaign} & \textbf{xsec (pb)} & \shortstack{\textbf{Filter} \\ \textbf{Efficiency}} & \shortstack{\textbf{Events} \\ \textbf{Generated}} & \shortstack{\textbf{Effective} \\  \textbf{Luminosity (\ifb)}} \\
    \hline
    \hline
    410470.PhPy8EG\_A14\_ttbar\_hdamp258p75\_nonallhad & MC16a & 729.77 & 0.54384 & 60413000 & 148.78 \\
    410471.PhPy8EG\_A14\_ttbar\_hdamp258p75\_allhad & MC16a & 729.78 & 0.45627 & 19994000 & 59.101 \\
    410472.PhPy8EG\_A14\_ttbar\_hdamp258p75\_dil & MC16a & 729.77 & 0.10546 & 39974000 & 511.18 \\
    \hline
    410470.PhPy8EG\_A14\_ttbar\_hdamp258p75\_nonallhad & MC16d & 729.77 & 0.54384 & 74486000 & 184.71 \\
    410471.PhPy8EG\_A14\_ttbar\_hdamp258p75\_allhad & MC16d & 729.78 & 0.45627 & 24714000 & 73.047 \\
    410472.PhPy8EG\_A14\_ttbar\_hdamp258p75\_dil & MC16d & 729.77 & 0.10546 & 44876000 & 573.88 \\
    \hline
    \hline
    410249.Sherpa\_221\_NNPDF30NNLO\_ttbar\_AllHadronic\_MEPS\_NLO & MC16a & 330.54 & 1.0 & 9993000 & 4.0608 \\
    410250.Sherpa\_221\_NNPDF30NNLO\_ttbar\_SingleLeptonP\_MEPS\_NLO & MC16a & 158.76 & 1.0 & 14987000 & 8.9811 \\
    410251.Sherpa\_221\_NNPDF30NNLO\_ttbar\_SingleLeptonM\_MEPS\_NLO & MC16a & 158.97 & 1.0 & 14991000 & 5.0114 \\
    410252.Sherpa\_221\_NNPDF30NNLO\_ttbar\_dilepton\_MEPS\_NLO & MC16a & 76.277 & 1.0 & 9995000 & 1.2000 \\
    \hline
    410249.Sherpa\_221\_NNPDF30NNLO\_ttbar\_AllHadronic\_MEPS\_NLO & MC16d & 330.44 & 1.0 & 9988000 & 2.6319 \\
    410250.Sherpa\_221\_NNPDF30NNLO\_ttbar\_SingleLeptonP\_MEPS\_NLO & MC16d & 158.74 & 1.0 & 14979000 & 15.336 \\
    410251.Sherpa\_221\_NNPDF30NNLO\_ttbar\_SingleLeptonM\_MEPS\_NLO & MC16d & 159.01 & 1.0 & 14986000 & 12.569 \\
    410252.Sherpa\_221\_NNPDF30NNLO\_ttbar\_dilepton\_MEPS\_NLO & MC16d & 76.09 & 1.0 & 9983000 & 0.48186 \\
  \end{tabular}}
  }
\caption{List of datasets used for $t\bar{t}$ Monte Carlo \cite{Krizka:2310645}.}
\label{table:data:ttbar}
\end{sidewaystable}

Single-top samples, containing a single top/anti-top quark and a
$W^{-}$/$W^{+}$, were generated with \textsc{Powheg-Box} 2~\cite{Campbell2012}
with the NNPDF30 parton distribution functions~\cite{Lai:2010vv}. This process
was showered using \textsc{Pythia} 8.230~\cite{Sjostrand:2014zea} configured
with the A14 tune and the NNPDF23 parton distribution
functions~\cite{Carrazza:2013axa} with all resulting $b$-hadrons decayed via
\textsc{EvtGen}~\cite{LANGE2001152}. The single-top samples used in this
analysis are summarized in \Cref{table:data:single_top}.

\begin{sidewaystable}
  \centerline{
  \centering
  {\tiny
  \begin{tabular}{l|c|cccc}
    \textbf{Dataset} & \textbf{Campaign} & \textbf{xsec (pb)} & \shortstack{\textbf{Filter} \\ \textbf{Efficiency}} & \shortstack{\textbf{Events} \\ \textbf{Generated}} & \shortstack{\textbf{Effective} \\  \textbf{Luminosity (\ifb)}} \\
    \hline
    \hline
    410646.PowhegPythia8EvtGen\_A14\_Wt\_DR\_inclusive\_top & MC16a & 37.936 & 1.0 & 4996000 & 130.45 \\
    410647.PowhegPythia8EvtGen\_A14\_Wt\_DR\_inclusive\_antitop & MC16a & 37.905 & 1.0 & 4999000 & 130.65 \\
    \hline
    410646.PowhegPythia8EvtGen\_A14\_Wt\_DR\_inclusive\_top & MC16d & 37.936 & 1.0 & 6243000 & 162.99 \\
    410647.PowhegPythia8EvtGen\_A14\_Wt\_DR\_inclusive\_antitop & MC16d & 37.905 & 1.0 & 6240000 & 163.10 \\
  \end{tabular}}
  }
\caption{List of datasets used for single-top Monte Carlo \cite{Krizka:2310645}.}
\label{table:data:single_top}
\end{sidewaystable}



