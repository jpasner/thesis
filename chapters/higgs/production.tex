\section{Higgs Production Mechanisms} \label{sec:higgs:production}

At the LHC the dominate production mechanisms for the higgs in order of
decreasing cross section are: gluon-fluon fusion (ggF), vector boson fusion
(VBF), vector boson associated production or ``Higgsstrahlung" (VH), and
associated production with $t\bar{t}$ ($t\bar{t}H$) and $b\bar{b}$
($b\bar{b}H$).  The cross sections with associated theoretical uncertainties for
each is shown as a function of the center of mass energy $\sqrt{s}$ in
\cref{fig:higgs_xsection} and the actual feynman diagrams can be seen in
\cref{fig:higgs_production}.

\begin{figure}[!htbp]
  \begin{center}
    \includegraphics[width=0.5\linewidth]{figures/higgs/higgs_xsection.pdf}
    \caption{ Cross section for the production of the SM Higgs boson as a
function of the center of mass energy ($\sqrt{s}$) at the LHC. \cite{PDG2018:Ch11}}
    \label{fig:higgs_xsection}
  \end{center}
\end{figure}

\begin{figure}[!htbp]
\centering

\subcaptionbox{ggF}{
\resizebox{0.48\textwidth}{!}{
\begin{tikzpicture}[thick]
 \begin{feynman}
  \vertex (origin);
  \vertex [right=1.5cm of origin] (H);
  \vertex [above left=1cm and 1.2cm of origin] (v1);
  \vertex [below left=1cm and 1.2cm of origin] (v2);
  \vertex [left=1.75cm of v1] (g1) {\(g\)};
  \vertex [left=1.75cm of v2] (g2) {\(g\)};
  \diagram* {
  (origin) -- [red, scalar, edge label={\(H\)}] (H),
  (origin) -- [fermion] (v1) -- [fermion] (v2) -- [fermion] (origin),
  (g1) -- [gluon] (v1),
  (g2) -- [gluon] (v2),
  };
 \end{feynman}
\end{tikzpicture}
}}
\subcaptionbox{VBF}{
\resizebox{0.48\textwidth}{!}{
\begin{tikzpicture}[thick]
 \begin{feynman}
  \vertex (origin);
  \vertex [right=1.5cm of origin] (H);
  \vertex [above left=1cm and 1.2cm of origin] (v1);
  \vertex [below left=1cm and 1.2cm of origin] (v2);
  \vertex [left=1.75cm of v1] (q1) {\(q\)};
  \vertex [left=1.75cm of v2] (q2) {\(q\)};
  \vertex [above right=0.25cm and 1.75cm of v1] (q3) {\(q'\)};
  \vertex [below right=0.25cm and 1.75cm of v2] (q4) {\(q'\)};
  \diagram* {
  (v1) -- [boson, edge label={\(V\)}] (origin),
  (origin) -- [boson, edge label={\(V\)}] (v2),
  (origin) -- [red, scalar, edge label={\(H\)}] (H),
  (q1) -- [fermion] (v1),
  (q2) -- [fermion] (v2),
  (v1) -- [fermion] (q3),
  (v2) -- [fermion] (q4),
  };
 \end{feynman}
\end{tikzpicture}
}} \\
\subcaptionbox{Higgsstrahlung}{
\resizebox{0.48\textwidth}{!}{
\begin{tikzpicture}[thick]
 \begin{feynman}
  \vertex (origin);
  \vertex [right=1.2cm of origin] (Vstar);
  \vertex [above right=0.50cm and 1.0cm of Vstar] (V1) {\(V\)};
  \vertex [red, below right=0.50cm and 1.0cm of Vstar] (H) {\(H\)};
  \vertex [above left=0.50cm and 1.0cm of origin] (g1) {\(\bar{q}\)};
  \vertex [below left=0.50cm and 1.0cm of origin] (g2) {\(q\)};
  \diagram* {
  (origin) -- [boson, edge label={\(V^{*}\)}] (Vstar) -- [boson] (V1),
  (Vstar) -- [red, scalar] (H),
  (origin) -- [fermion] (g1),
  (g2) -- [fermion] (origin),
  };
 \end{feynman}
\end{tikzpicture}
}}
\subcaptionbox{$t\bar{t}$ ($t\bar{t}H$) and $b\bar{b}$ ($b\bar{b}H$)}{
\resizebox{0.48\textwidth}{!}{
\begin{tikzpicture}[thick]
 \begin{feynman}
  \vertex (origin);
  \vertex [right=1.5cm of origin] (H);
  \vertex [above left=1cm and 0.2cm of origin] (v1);
  \vertex [below left=1cm and 0.2cm of origin] (v2);
  \vertex [left=1.75cm of v1] (g1) {\(g\)};
  \vertex [left=1.75cm of v2] (g2) {\(g\)};
  \vertex [above right=1.2cm and 1.4cm of origin] (t1) {\(t,b\)};
  \vertex [below right=1.2cm and 1.4cm of origin] (t2) {\(\bar{t},\bar{b}\)};
  \diagram* {
  (g1) -- [gluon] (v1),
  (g2) -- [gluon] (v2),
  (origin) -- [red, scalar, edge label={\(H\)}] (H),
  (origin) -- [fermion] (v1) -- [fermion] (t1),
  (t2) -- [fermion] (v2) -- [fermion] (origin),
  };
 \end{feynman}
\end{tikzpicture}
}}

\caption{Feynman diagrams representing the dominant Higgs production modes at
the LHC.}

\label{fig:higgs_production}
\end{figure}

The dominant Higgs production mechanism at hadron colliders is ggF.  This may
seem strange as gluons are massless and thus do not couple directly to the
Higgs.  Instead the gluons indirectly couple to the Higgs via a quark loop.  As
discussed in \cref{sec:theory:fermion_mass}, the coupling of a fermion
is proportional to $m_f$ so the dominant contribution to this quark loop comes
from the top quark.  

The second largest cross section for Higgs production at the LHC comes from the VBF
mechanism.  In VBF the initial state quarks scatter via the exchange of a
$W^{\pm}$ or $Z$ boson which subsequently radiates the Higgs boson.  Unlike ggF
this production mechanism scatters the initial state quarks which allows them to
be observed as part of the interaction.  The existence of these extra quarks
makes these interactions easier to select for during analysis.

Next we have Higgs production in association with a vector boson. The cross
section for this is even lower than the above two, but remains important due to
the easily selected signature of the decaying vector boson.  The largest
background at the LHC is multijet events coming from interactions that
produce strong force objects.  Thus the leptons from the boson's decay act as a
discriminator from this multijet background greatly reducing its effect on
sensitivity.

With the lowest cross section of the four methods discussed we have the production of
the Higgs in assocaiation with either $b\bar{b}$ or $t\bar{t}$.  This channel is
important due to our ability to measure not only the Higgs, but also the quarks
that it directly coupled with.  This allowes us to directly measure the coupling of
the Higgs to that quark, unlike the ggF method where the quark in the loop is
never directly observed.

As we can see, each of these methods has its advantages and disadvantages as
well as different valuable information that can be extracted.  The result is a
need for many different analysis using different techniques to search for each
mechanism.
