\section{Boosted Higgs} \label{sec:higgs:boosted}

The strong agreement between the theoretical predictions of the SM Higgs boson
and experiment shown in \Cref{sec:higgs:higgs_evidence} represents the
fulfillment of a generation of incredible technological and theoretical
achievement.  The next step is to push the search for deviations from the model
that might hint at the physics of mysteries like the matter~/~anti-matter
asymmetry of the universe, dark matter, the particle nature of gravity and dark
energy. 

Effective field theory arguments for extensions of the SM suggest that precise
measurments of the shape of the momentum distribution for highly boosted (high
momentum) Higgs offer the opportunity to see the effects of natural new
physics.  In particular, at high enough energies $-$ $p_{T,H} \geq \text{500
GeV}$ $-$ the ggF production mode of the Higgs becomes sensitive to the heavy
heavy fermion loop \cite{Schlaffer:2014osa}.  New resonances that run in the
loop would contribute to the coupling strength of the effective
gluon-gluon-Higgs interaction and would give an anomalous result compared to
the SM. In references \cite{Schlaffer:2014osa,Grojean:2013nya,Dawson:2015gka}
the effect on the production cross section for boosted Higgs through ggF can
exceed 50\% for $p_{T,H} \geq \text{500 GeV}$.

Searches for boosted Higgs are also made easier as the LHC produces a large
number of soft (low momentum) QCD interactions as seen in
\Cref{fig:xsection_measurements}.  Thus a boosted signal is easier to
differentiate from the common QCD interactions which fall off exponentially as
a function of momentum.  However, to achieve this boost, the Higgs must recoil
off of a high energy jet or photon \cite{Aaboud:2018zba} produced through
initial state radiation (ISR) as seen in \Cref{fig:Hbb_ISR} with $H \rightarrow
b\bar{b}$. In this thesis only strongly produced ISR is considered in order to
simplify the analysis.

\begin{figure}[!htbp]
\centering
\begin{tikzpicture}[thick]
 \begin{feynman}
  \vertex (origin);
  \vertex [right=1.5cm of origin] (H);
  \vertex [above right=0.50cm and 1.5cm of H] (b1) {\(\bar{b}\)};
  \vertex [below right=0.50cm and 1.5cm of H] (b2) {\(b\)};
  \vertex [above left=1cm and 1.2cm of origin] (v1);
  \vertex [below left=1cm and 1.2cm of origin] (v2);
  \vertex [left=1.75cm of v1] (g1) {\(g\)};
  \vertex [left=1.75cm of v2] (g2) {\(g\)};
  \vertex at ($(g1)!0.5!(v1)$) (ISR_start);
  \vertex [above right=0.8cm and 1.5 cm of ISR_start] (ISR) {ISR};
  \diagram* {
  (origin) -- [red, scalar, edge label={\(H\)}] (H) -- [anti fermion] (b1),
  (H) -- [fermion] (b2),
  (origin) -- [fermion] (v1) -- [fermion] (v2) -- [fermion] (origin),
  (g1) -- [gluon] (v1),
  (g2) -- [gluon] (v2),
  (ISR_start) -- [gluon] (ISR),
  };
 \end{feynman}
\end{tikzpicture}
\caption{Feynman diagram for boosted Higgs decaying to $b\bar{b}$}
\label{fig:Hbb_ISR}
\end{figure}

As a result of this boost, the decay products of the Higgs and the
hadronization products of the ISR become highly collimated as shown in
\Cref{fig:Hbb_ISR_FatJets}.  The two pronged structure of the jet that results
from the $H \rightarrow b\bar{b}$ decay provides a unique signature that can be
used to differentiate the Higgs signal from other QCD process.

% c.f. https://tex.stackexchange.com/a/298288/152544
\tikzset{
 pics/fatjet/.style args={#1}{
  % default rotation: 0 degrees (jet progressing to right)
  code={
   \draw [fill=#1!30, thick,join=round](0, 0) -- (4, -1) -- (4, 1) --cycle;
   \draw [fill=#1!30, thick](4, 0) ellipse (.25 and 1);
  }
 }
}

\begin{figure}[!htbp]
\centering
\resizebox{0.99\textwidth}{!}{
\begin{tikzpicture}[thick]
 \path (0, 0)  pic {fatjet=blue};
 \path (0, 0)  pic [rotate=-180] {fatjet=green};

 \begin{feynman}
  \vertex [dot] (origin) {};
  \vertex [right=1cm of origin, label={[red, xshift=-0.5cm, yshift=0.15cm]\(H\)}] (H);
  \vertex [above right=0.20cm and 2.3cm of H] (b1) {\(\bar{b}\)};
  \vertex [below right=0.20cm and 2.3cm of H] (b2) {\(b\)};
  \vertex [left=1cm of origin, label={[xshift=0.5cm, yshift=0.15cm]\(g\)}] (jet);
  \vertex [above left=0.20cm and 2.3cm of jet] (q1);
  \vertex [below left=0.20cm and 2.3cm of jet] (q2);
  \vertex [above left=0.40cm and 2.5cm of jet] (q3);
  \vertex [below left=0.40cm and 2.1cm of jet] (q4);
  \vertex [above left=0.05cm and 2.0cm of jet] (q5);
  \vertex [right=5cm of origin] (largeR_signal);
  \vertex [left=5cm of origin] (largeR_jet);
  \diagram* {
  (origin) -- [red, scalar, momentum'={}, style={pos=0.2}] (H) -- [anti fermion] (b1),
  (H) -- [fermion] (b2),
  (origin) -- [gluon, momentum={}] (jet),
  (jet) -- (q1),
  (jet) -- (q2),
  (jet) -- (q3),
  (jet) -- (q4),
  (jet) -- (q5),
  };
 \end{feynman}
\end{tikzpicture}
}
\caption{Cartoon showing columnated Higgs and ISR as a result of the large
boost due to their mutual recoil.} 
\label{fig:Hbb_ISR_FatJets}
\end{figure}

In pursuit of the rich physics potential discussed above an analysis of Boosted
Higgs signatures was undertaken and is discussed further in \Cref{part:HbbISR}.
