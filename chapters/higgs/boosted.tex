\section{Boosted Higgs} \label{sec:higgs:boosted}

The strong agreement between the theoretical predictions for the SM Higgs boson
and the experiment shown in \Cref{sec:higgs:higgs_evidence} represents the
fulfillment of a generation of incredible technological and theoretical
achievement.  The next step is to complete the set of Higgs coupling
measurements, like adding ggF $H \rightarrow b\bar{b}$, and to push the search
for deviations from the Standard Model that might hint at the new physics of
mysteries like the matter~/~anti-matter asymmetry of the universe, dark matter,
the particle nature of gravity, and dark energy. 

Minimal Supersymmetric Standard Model (MSSM) and Standard Model Effective Field
Theory (SMEFT) arguments for extensions of the SM suggest that precise
measurements of the shape of the momentum distribution for highly boosted (high
momentum) Higgs bosons offer the opportunity to see the effects of new physics.
For example, in MSSM the cross section for ggF production of Higgs bosons with
$p_{\text{T},H} \geq \text{500 GeV}$ could be enhanced or suppressed as a
result of new heavy particles participating in the top quark loop
\cite{Schlaffer:2014osa}. The effect of these new particles on the observed
Higgs boson $\pT$ spectrum would be limited to the high momentum region where
there is enough energy in the quark loop to allow these heavy particles to
contribute significantly. Furthermore,  SMEFT models predict modifications to
the high $\pT$ spectrum for boosted Higgs bosons as a result of new physics
effects that would modify the top and bottom Yukawa couplings.  These new
physics effects are parameterized by Wilson coefficients which can vary the
predicted cross section, and thus the predicted $\pT$ spectrum, up and down
while maintaining agreement with current experimental results
\cite{Grazzini:2016paz}. Both of these examples would modify the effective ggH
interaction and thus give an anomalous result compared to the SM. The effect on
the production cross section for boosted Higgs through ggF can exceed 50\% for
$p_{\text{T},H} \geq \text{500 GeV}$
\cite{Schlaffer:2014osa,Grojean:2013nya,Grazzini:2016paz,Dawson:2015gka}.

Searches for boosted Higgs bosons are also easier because the LHC produces a
large number of soft (low momentum) QCD interactions as seen in
\Cref{fig:xsection_measurements}.  Because of this a boosted signal is easier
to differentiate from the common QCD interactions which fall off exponentially
as a function of momentum.  However, to achieve this boost, the Higgs must
recoil off of a high energy jet or photon \cite{Aaboud:2018zba} produced
through initial state radiation (ISR). This dissertation only considers
strongly produced ISR due to the low cross section for the $H + \gamma$ final
state.  Thus, the analysis presented in \Cref{part:HbbISR} searches for a final
state signature of $H$ + jet in the $H \rightarrow b\bar{b}$ decay mode.  An
example leading order Feynman diagram for this process is shown in
\Cref{fig:Hbb_ISR}.

\begin{figure}[!htbp]
\centering
\begin{tikzpicture}[thick]
 \begin{feynman}
  \vertex (origin);
  \vertex [right=1.5cm of origin] (H);
  \vertex [above right=0.50cm and 1.5cm of H] (b1) {\(\bar{b}\)};
  \vertex [below right=0.50cm and 1.5cm of H] (b2) {\(b\)};
  \vertex [above left=1cm and 1.2cm of origin] (v1);
  \vertex [below left=1cm and 1.2cm of origin] (v2);
  \vertex [left=1.75cm of v1] (g1) {\(g\)};
  \vertex [left=1.75cm of v2] (g2) {\(g\)};
  \vertex at ($(g1)!0.5!(v1)$) (ISR_start);
  \vertex [above right=0.8cm and 1.5 cm of ISR_start] (ISR) {ISR};
  \diagram* {
  (origin) -- [red, scalar, edge label={\(H\)}] (H) -- [anti fermion] (b1),
  (H) -- [fermion] (b2),
  (origin) -- [fermion] (v1) -- [fermion] (v2) -- [fermion] (origin),
  (g1) -- [gluon] (v1),
  (g2) -- [gluon] (v2),
  (ISR_start) -- [gluon] (ISR),
  };
 \end{feynman}
\end{tikzpicture}
\caption{Feynman diagram for boosted Higgs decaying to $b\bar{b}$}
\label{fig:Hbb_ISR}
\end{figure}

As a result of this boost, the decay products of the Higgs boson and the
hadronization products of the ISR become highly collimated as shown in
\Cref{fig:Hbb_ISR_FatJets}.  The two pronged structure of the jet that results
from the $H \rightarrow b\bar{b}$ decay provides a unique signature that can be
used to differentiate the Higgs signal from other QCD processes.

% c.f. https://tex.stackexchange.com/a/298288/152544
\tikzset{
 pics/fatjet/.style args={#1}{
  % default rotation: 0 degrees (jet progressing to right)
  code={
   \draw [fill=#1!30, thick,join=round](0, 0) -- (4, -1) -- (4, 1) --cycle;
   \draw [fill=#1!30, thick](4, 0) ellipse (.25 and 1);
  }
 }
}

\begin{figure}[!htbp]
\centering
\resizebox{0.99\textwidth}{!}{
\begin{tikzpicture}[thick]
 \path (0, 0)  pic {fatjet=blue};
 \path (0, 0)  pic [rotate=-180] {fatjet=green};

 \begin{feynman}
  \vertex [dot] (origin) {};
  \vertex [right=1cm of origin, label={[red, xshift=-0.5cm, yshift=0.15cm]\(H\)}] (H);
  \vertex [above right=0.20cm and 2.3cm of H] (b1) {\(\bar{b}\)};
  \vertex [below right=0.20cm and 2.3cm of H] (b2) {\(b\)};
  \vertex [left=1cm of origin, label={[xshift=0.5cm, yshift=0.15cm]\(g\)}] (jet);
  \vertex [above left=0.20cm and 2.3cm of jet] (q1);
  \vertex [below left=0.20cm and 2.3cm of jet] (q2);
  \vertex [above left=0.40cm and 2.5cm of jet] (q3);
  \vertex [below left=0.40cm and 2.1cm of jet] (q4);
  \vertex [above left=0.05cm and 2.0cm of jet] (q5);
  \vertex [right=5cm of origin] (largeR_signal);
  \vertex [left=5cm of origin] (largeR_jet);
  \diagram* {
  (origin) -- [red, scalar, momentum'={}, style={pos=0.2}] (H) -- [anti fermion] (b1),
  (H) -- [fermion] (b2),
  (origin) -- [gluon, momentum={}] (jet),
  (jet) -- (q1),
  (jet) -- (q2),
  (jet) -- (q3),
  (jet) -- (q4),
  (jet) -- (q5),
  };
 \end{feynman}
\end{tikzpicture}
}
\caption{Cartoon showing collimated Higgs boson jet and ISR jet as a result of the large
boost due to their mutual recoil \cite{Feickert:2690521}.} 
\label{fig:Hbb_ISR_FatJets}
\end{figure}

In pursuit of the rich physics potential discussed above, an analysis of a
boosted Higgs decaying to a $b\bar{b}$ pair with an associated ISR jet was
undertaken and is discussed further in \Cref{part:HbbISR}. However, before
giving more details about the ATLAS data analysis, it will be useful to
describe the experimental apparatus and reconstruction methods.
