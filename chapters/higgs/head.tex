\chapter{Boosted Higgs at the LHC} \label{chap:higgs}

In \cref{chap:standard_model} I've shown how the higgs mechanism resolves
inconsistencies of the model surrounding the generation of gauge boson and
fermion mass terms while also maintaining gauge invariance.  However to
understand the search for and resulting discovery of this SM Higgs boson
requires the discussion of how one goes about producing and detecting the
physical object itself.  In order to gather sufficient statistics to validate
the theory we require a collider capable of putting enough energy into a
collision to rapidly produce Higgs bosons for study.  To this end the Large
Hadron Collider (LHC) discussed in \cref{chap:lhc} was laboriously
designed, funded, and constructed by the largest international collaboration of
scientists on the planet. In this chapter I will discuss the relevant Higgs
boson production mechanisms available at the LHC as well as the various decay
modes of the Higgs that were used for its discovery, and are currently used to
measure its properties.

\section{Higgs Production Mechanisms} \label{sec:higgs:production}

At the LHC the dominate production mechanisms for the higgs in order of
decreasing cross section are: gluon-fluon fusion (ggF), vector boson fusion
(VBF), vector boson associated production or ``Higgsstrahlung" (VH), and
associated production with $t\bar{t}$ ($t\bar{t}H$) and $b\bar{b}$
($b\bar{b}H$).  The cross sections with associated theoretical uncertainties for
each is shown as a function of the center of mass energy $\sqrt{s}$ in
\cref{fig:higgs_xsection} and the actual feynman diagrams can be seen in
\cref{fig:higgs_production}.

\begin{figure}[!htbp]
  \begin{center}
    \includegraphics[width=0.5\linewidth]{figures/higgs/higgs_xsection.pdf}
    \caption{ Cross section for the production of the SM Higgs boson as a
function of the center of mass energy ($\sqrt{s}$) at the LHC. \cite{PDG2018:Ch11}}
    \label{fig:higgs_xsection}
  \end{center}
\end{figure}

\begin{figure}[!htbp]
\centering

\subcaptionbox{ggF}{
\resizebox{0.48\textwidth}{!}{
\begin{tikzpicture}[thick]
 \begin{feynman}
  \vertex (origin);
  \vertex [right=1.5cm of origin] (H);
  \vertex [above left=1cm and 1.2cm of origin] (v1);
  \vertex [below left=1cm and 1.2cm of origin] (v2);
  \vertex [left=1.75cm of v1] (g1) {\(g\)};
  \vertex [left=1.75cm of v2] (g2) {\(g\)};
  \diagram* {
  (origin) -- [red, scalar, edge label={\(H\)}] (H),
  (origin) -- [fermion] (v1) -- [fermion] (v2) -- [fermion] (origin),
  (g1) -- [gluon] (v1),
  (g2) -- [gluon] (v2),
  };
 \end{feynman}
\end{tikzpicture}
}}
\subcaptionbox{VBF}{
\resizebox{0.48\textwidth}{!}{
\begin{tikzpicture}[thick]
 \begin{feynman}
  \vertex (origin);
  \vertex [right=1.5cm of origin] (H);
  \vertex [above left=1cm and 1.2cm of origin] (v1);
  \vertex [below left=1cm and 1.2cm of origin] (v2);
  \vertex [left=1.75cm of v1] (q1) {\(q\)};
  \vertex [left=1.75cm of v2] (q2) {\(q\)};
  \vertex [above right=0.25cm and 1.75cm of v1] (q3) {\(q'\)};
  \vertex [below right=0.25cm and 1.75cm of v2] (q4) {\(q'\)};
  \diagram* {
  (v1) -- [boson, edge label={\(V\)}] (origin),
  (origin) -- [boson, edge label={\(V\)}] (v2),
  (origin) -- [red, scalar, edge label={\(H\)}] (H),
  (q1) -- [fermion] (v1),
  (q2) -- [fermion] (v2),
  (v1) -- [fermion] (q3),
  (v2) -- [fermion] (q4),
  };
 \end{feynman}
\end{tikzpicture}
}} \\
\subcaptionbox{Higgsstrahlung}{
\resizebox{0.48\textwidth}{!}{
\begin{tikzpicture}[thick]
 \begin{feynman}
  \vertex (origin);
  \vertex [right=1.2cm of origin] (Vstar);
  \vertex [above right=0.50cm and 1.0cm of Vstar] (V1) {\(V\)};
  \vertex [red, below right=0.50cm and 1.0cm of Vstar] (H) {\(H\)};
  \vertex [above left=0.50cm and 1.0cm of origin] (g1) {\(\bar{q}\)};
  \vertex [below left=0.50cm and 1.0cm of origin] (g2) {\(q\)};
  \diagram* {
  (origin) -- [boson, edge label={\(V^{*}\)}] (Vstar) -- [boson] (V1),
  (Vstar) -- [red, scalar] (H),
  (origin) -- [fermion] (g1),
  (g2) -- [fermion] (origin),
  };
 \end{feynman}
\end{tikzpicture}
}}
\subcaptionbox{$t\bar{t}$ ($t\bar{t}H$) and $b\bar{b}$ ($b\bar{b}H$)}{
\resizebox{0.48\textwidth}{!}{
\begin{tikzpicture}[thick]
 \begin{feynman}
  \vertex (origin);
  \vertex [right=1.5cm of origin] (H);
  \vertex [above left=1cm and 0.2cm of origin] (v1);
  \vertex [below left=1cm and 0.2cm of origin] (v2);
  \vertex [left=1.75cm of v1] (g1) {\(g\)};
  \vertex [left=1.75cm of v2] (g2) {\(g\)};
  \vertex [above right=1.2cm and 1.4cm of origin] (t1) {\(t,b\)};
  \vertex [below right=1.2cm and 1.4cm of origin] (t2) {\(\bar{t},\bar{b}\)};
  \diagram* {
  (g1) -- [gluon] (v1),
  (g2) -- [gluon] (v2),
  (origin) -- [red, scalar, edge label={\(H\)}] (H),
  (origin) -- [fermion] (v1) -- [fermion] (t1),
  (t2) -- [fermion] (v2) -- [fermion] (origin),
  };
 \end{feynman}
\end{tikzpicture}
}}

\caption{Feynman diagrams representing the dominant Higgs production modes at
the LHC.}

\label{fig:higgs_production}
\end{figure}

The dominant Higgs production mechanism at hadron colliders is ggF.  This may
seem strange as gluons are massless and thus do not couple directly to the
Higgs.  Instead the gluons indirectly couple to the Higgs via a quark loop.  As
discussed in \cref{sec:theory:fermion_mass}, the coupling of a fermion
is proportional to $m_f$ so the dominant contribution to this quark loop comes
from the top quark.  

The second largest cross section for Higgs production at the LHC comes from the VBF
mechanism.  In VBF the initial state quarks scatter via the exchange of a
$W^{\pm}$ or $Z$ boson which subsequently radiates the Higgs boson.  Unlike ggF
this production mechanism scatters the initial state quarks which allows them to
be observed as part of the interaction.  The existence of these extra quarks
makes these interactions easier to select for during analysis.

Next we have Higgs production in association with a vector boson. The cross
section for this is even lower than the above two, but remains important due to
the easily selected signature of the decaying vector boson.  The largest
background at the LHC is multijet events coming from interactions that
produce strong force objects.  Thus the leptons from the boson's decay act as a
discriminator from this multijet background greatly reducing its effect on
sensitivity.

With the lowest cross section of the four methods discussed we have the production of
the Higgs in assocaiation with either $b\bar{b}$ or $t\bar{t}$.  This channel is
important due to our ability to measure not only the Higgs, but also the quarks
that it directly coupled with.  This allowes us to directly measure the coupling of
the Higgs to that quark, unlike the ggF method where the quark in the loop is
never directly observed.

As we can see, each of these methods has its advantages and disadvantages as
well as different valuable information that can be extracted.  The result is a
need for many different analysis using different techniques to search for each
mechanism.

\section{Parton Distribution Function} \label{sec:higgs:partons}

The LHC collides protons, however looking at the Feynman diagrams in
\Cref{fig:higgs_production} it is quarks and gluons (a.k.a partons) that
produce these fundamental interactions. This is an indicator that when the
production cross section is calculated for a process at the LHC, one must not
only consider the hard-scatter probability of the specific diagram, but also
consider the composition of the proton itself.  Furthermore, the calculation
must consider the fraction of the total momentum of the proton held by each of
its constituent partons.  This concept is described by Parton Distribution
Functions (PDFs) which give the probability that the indicated parton carries
momentum fraction $x$ of the proton when probed at energy scale $Q$.  An
example PDF for $Q = 10~\GeV^2$ and $Q = 10^{4}~\GeV$ is shown in
\Cref{fig:parton_distribution_function}.

\begin{figure}[!htbp]
  \begin{center}
    \includegraphics[width=0.8\linewidth]{figures/higgs/pdf.pdf}
    \caption{MMHT2014 NNLO PDFs at $Q^2 = 10~\GeV^{2}$ and $Q^2
=10^{4}~\GeV^{2}$ with associated 68\% confidence-level uncertainty bands
\cite{Harland-Lang2015}.  The colored regions indicate the probability of
finding the labeled parton with a momentum fraction given along the $x$ axis.
As expected the valence quarks contain the largest fraction of the momentum
while the gluons are more likely to carry smaller fractions of the total
momentum.  Note that as $Q^2$ increases the contributions from sea quarks
increases.}
    \label{fig:parton_distribution_function}
  \end{center}
\end{figure}

\section{Branching Ratios} \label{sec:higgs:branching}

The coupling of the SM Higgs with the gauge bosons and fermions has been shown
to give these particles their mass, however it also means that the Higgs can
decay into all of these particles.  In order of most to least likely final
states of a Higgs decay we have the decay to; a pair of $b$-quarks ($b\bar{b}$),
a pair of weak vector bosons where one is off-shell ($VV^{*}$), two gluons
($gg$), a duo of tau leptons ($\tau^{+}\tau^{-}$), or a pair of photons
($\gamma\gamma$).  Similar to the ggF production mechanism discussed in
\cref{sec:higgs:production} the decays to massless gauge bosons (photons and
gluons) are facilitated through loops of massive particles. The exact feynman
diagrams depicting the above process' are shown in \cref{fig:higgs_decay} while
information about their branching ratios is detailed in
\cref{table:higgs_branching_ratios}.

\begin{figure}[!htbp]
\centering

\subcaptionbox{$H \rightarrow b\bar{b}$}{
\resizebox{0.48\textwidth}{!}{
\begin{tikzpicture}[thick]
 \begin{feynman}
  \vertex (origin);
  \vertex [right=1.5cm of origin] (H);
  \vertex [above right=0.50cm and 1.5cm of H] (b1) {\(\bar{b}\)};
  \vertex [below right=0.50cm and 1.5cm of H] (b2) {\(b\)};
  \diagram* {
  (origin) -- [red, scalar, edge label={\(H\)}] (H),
  (b1) -- [fermion] (H) -- [fermion] (b2),
  };
 \end{feynman}
\end{tikzpicture}
}}
\subcaptionbox{$H \rightarrow W^{\pm}W^{\mp*}$}{
\resizebox{0.48\textwidth}{!}{
\begin{tikzpicture}[thick]
 \begin{feynman}
  \vertex (origin);
  \vertex [right=1.5cm of origin] (H);
  \vertex [above right=0.50cm and 1.5cm of H] (W1) {\(W^{\pm}\)};
  \vertex [below right=0.50cm and 1.5cm of H] (W2) {\({W^{\mp}}^{*}\)};
  \diagram* {
  (origin) -- [red, scalar, edge label={\(H\)}] (H),
  (W1) -- [boson] (H),
  (W2) -- [boson] (H),
  };
 \end{feynman}
\end{tikzpicture}
}} \\
\subcaptionbox{$H \rightarrow gg$}{
\resizebox{0.48\textwidth}{!}{
\begin{tikzpicture}[thick]
 \begin{feynman}
  \vertex (origin);
  \vertex [right=1.5cm of origin] (H);
  \vertex [above right=1cm and 1.2cm of H] (v1);
  \vertex [below right=1cm and 1.2cm of H] (v2);
  \vertex [right=1.75cm of v1] (g1) {\(g\)};
  \vertex [right=1.75cm of v2] (g2) {\(g\)};
  \diagram* {
  (origin) -- [red, scalar, edge label={\(H\)}] (H),
  (H) -- [fermion] (v1) -- [fermion] (v2) -- [fermion] (H),
  (g1) -- [gluon] (v1),
  (g2) -- [gluon] (v2),
  };
 \end{feynman}
\end{tikzpicture}
}}
\subcaptionbox{$H \rightarrow \tau^{+}\tau{-}$}{
\resizebox{0.48\textwidth}{!}{
\begin{tikzpicture}[thick]
 \begin{feynman}
  \vertex (origin);
  \vertex [right=1.5cm of origin] (H);
  \vertex [above right=0.50cm and 1.5cm of H] (tau1) {\(\tau^{+}\)};
  \vertex [below right=0.50cm and 1.5cm of H] (tau2) {\(\tau^{-}\)};
  \diagram* {
  (origin) -- [red, scalar, edge label={\(H\)}] (H),
  (tau1) -- [fermion] (H) -- [fermion] (tau2),
  };
 \end{feynman}
\end{tikzpicture}
}} \\
\subcaptionbox{$H \rightarrow ZZ^{*}$}{
\resizebox{0.48\textwidth}{!}{
\begin{tikzpicture}[thick]
 \begin{feynman}
  \vertex (origin);
  \vertex [right=1.5cm of origin] (H);
  \vertex [above right=0.50cm and 1.5cm of H] (Z1) {\(Z\)};
  \vertex [below right=0.50cm and 1.5cm of H] (Z2) {\(Z^{*}\)};
  \diagram* {
  (origin) -- [red, scalar, edge label={\(H\)}] (H),
  (Z1) -- [boson] (H),
  (Z2) -- [boson] (H),
  };
 \end{feynman}
\end{tikzpicture}
}}
\subcaptionbox{$H \rightarrow \gamma\gamma$}{
\resizebox{0.48\textwidth}{!}{
\begin{tikzpicture}[thick]
 \begin{feynman}
  \vertex (origin);
  \vertex [right=1.5cm of origin] (H);
  \vertex [above right=1cm and 1.2cm of H] (v1);
  \vertex [below right=1cm and 1.2cm of H] (v2);
  \vertex [right=1.75cm of v1] (gamma1) {\(\gamma\)};
  \vertex [right=1.75cm of v2] (gamma2) {\(\gamma\)};
  \diagram* {
  (origin) -- [red, scalar, edge label={\(H\)}] (H),
  (H) -- [fermion] (v1) -- [fermion] (v2) -- [fermion] (H),
  (gamma1) -- [photon] (v1),
  (gamma2) -- [photon] (v2),
  };
 \end{feynman}
\end{tikzpicture}
}}


\caption{Feynman diagrams representing the leading Higgs decay channels.}

\label{fig:higgs_decay}
\end{figure}

\begin{table}[htpb]
 \centering
 \caption{ The branching ratios and the relative uncertainty for a Standard Model Higgs boson with $m_{H}=125~\GeV$ \cite{PDG2018:Ch11}.}
 \begin{tabular}{@{}lrr@{}} \toprule
  Decay Channel           & Branching Ratio       & Relative Uncertainty \\ \midrule
  $H\to b\bar{b}$         & $5.84 \times 10^{-1}$ & $_{-3.3\%}^{+3.2\%}$ \\
  \addlinespace[0.3em]
  $H\to W^{+}W^{-}$       & $2.14 \times 10^{-1}$ & $_{-4.2\%}^{+4.3\%}$ \\
  \addlinespace[0.3em]
  $H\to \tau^{+}\tau^{-}$ & $6.27 \times 10^{-2}$ & $_{-5.7\%}^{+5.7\%}$ \\
  \addlinespace[0.3em]
  $H\to ZZ$               & $2.62 \times 10^{-2}$ & $_{-4.1\%}^{+4.3\%}$ \\
  \addlinespace[0.3em]
  $H\to \gamma\gamma$     & $2.27 \times 10^{-3}$ & $_{-4.9\%}^{+5.0\%}$ \\
  \addlinespace[0.3em]
  $H\to Z\gamma$          & $1.53 \times 10^{-3}$ & $_{-8.9\%}^{+9.0\%}$ \\
  \addlinespace[0.3em]
  $H\to \mu^{+}\mu^{-}$   & $2.18 \times 10^{-4}$ & $_{-5.9\%}^{+6.0\%}$ \\
  \bottomrule
 \end{tabular}\label{table:higgs_branching_ratios}
\end{table} 

In \cref{table:higgs_branching_ratios} the order is determined by two distinct effects; 
the proportionality of the Higgs couplings to the mass of the decay product, and whether 
or not the rest mass of the higgs is sufficient to produce the two final state objects.  
In \cref{fig:higgs_decay_plot} you can see that as the mass of the higgs boson gets 
closer to $2m_W$ the cross section for $H \rightarrow WW$ grows.

\begin{figure}[!htbp]
  \begin{center}
    \includegraphics[width=0.5\linewidth]{figures/higgs/higgs_decay_plot.eps}
    \caption{Branching ratios for the decay of the SM Higgs boson near $m_{H} = 125$GeV including theoretical uncertainty bands \cite{PDG2018:Ch11}}
    \label{fig:higgs_decay_plot}
  \end{center}
\end{figure}



\section{Discovery} \label{sec:higgs:discovery}

\section{Boosted Higgs} \label{sec:higgs:boosted}

The strong agreement between the theoretical predictions of the SM Higgs boson
and experiment shown in \Cref{sec:higgs:higgs_evidence} represents the
fulfillment of a generation of incredible technological and theoretical
achievement.  The next step is to push the search for deviations from the model
that might hint at the physics of mysteries like the matter~/~anti-matter
asymmetry of the universe, dark matter, the particle nature of gravity and dark
energy. 

Effective field theory arguments for extensions of the SM suggest that precise
measurments of the shape of the momentum distribution for highly boosted (high
momentum) Higgs offer the opportunity to see the effects of natural new
physics.  In particular, at high enough energies $-$ $p_{T,H} \geq \text{500
GeV}$ $-$ the ggF production mode of the Higgs becomes sensitive to the heavy
heavy fermion loop \cite{Schlaffer:2014osa}.  New resonances that run in the
loop would contribute to the coupling strength of the effective
gluon-gluon-Higgs interaction and would give an anomalous result compared to
the SM. In references \cite{Schlaffer:2014osa,Grojean:2013nya,Dawson:2015gka}
the effect on the production cross section for boosted Higgs through ggF can
exceed 50\% for $p_{T,H} \geq \text{500 GeV}$.

Searches for boosted Higgs are also made easier as the LHC produces a large
number of soft (low momentum) QCD interactions.  Thus a boosted signal is
easier to differentiate from the common QCD interactions which fall off
exponentially as a function of momentum.  However, to achieve this boost, the
Higgs must recoil off of a high energy jet or photon \cite{Aaboud:2018zba}
produced through initial state radiation (ISR) as seen in \Cref{fig:Hbb_ISR}
with $H \rightarrow b\bar{b}$. In this thesis only strongly produced ISR is
considered in order to simplify the analysis.

\begin{figure}[!htbp]
\centering
\begin{tikzpicture}[thick]
 \begin{feynman}
  \vertex (origin);
  \vertex [right=1.5cm of origin] (H);
  \vertex [above right=0.50cm and 1.5cm of H] (b1) {\(\bar{b}\)};
  \vertex [below right=0.50cm and 1.5cm of H] (b2) {\(b\)};
  \vertex [above left=1cm and 1.2cm of origin] (v1);
  \vertex [below left=1cm and 1.2cm of origin] (v2);
  \vertex [left=1.75cm of v1] (g1) {\(g\)};
  \vertex [left=1.75cm of v2] (g2) {\(g\)};
  \vertex at ($(g1)!0.5!(v1)$) (ISR_start);
  \vertex [above right=0.8cm and 1.5 cm of ISR_start] (ISR) {ISR};
  \diagram* {
  (origin) -- [red, scalar, edge label={\(H\)}] (H) -- [anti fermion] (b1),
  (H) -- [fermion] (b2),
  (origin) -- [fermion] (v1) -- [fermion] (v2) -- [fermion] (origin),
  (g1) -- [gluon] (v1),
  (g2) -- [gluon] (v2),
  (ISR_start) -- [gluon] (ISR),
  };
 \end{feynman}
\end{tikzpicture}
\caption{Feynman diagram for boosted Higgs decaying to $b\bar{b}$}
\label{fig:Hbb_ISR}
\end{figure}

As a result of this boost, the decay products of the Higgs and the
hadronization products of the ISR become highly collimated as shown in
\Cref{fig:Hbb_ISR_FatJets}.  The two pronged structure of the jet that results
from the $H \rightarrow b\bar{b}$ decay provides a unique signature that can be
used to differentiate the Higgs signal from other QCD process.

% c.f. https://tex.stackexchange.com/a/298288/152544
\tikzset{
 pics/fatjet/.style args={#1}{
  % default rotation: 0 degrees (jet progressing to right)
  code={
   \draw [fill=#1!30, thick,join=round](0, 0) -- (4, -1) -- (4, 1) --cycle;
   \draw [fill=#1!30, thick](4, 0) ellipse (.25 and 1);
  }
 }
}

\begin{figure}[!htbp]
\centering
\resizebox{0.99\textwidth}{!}{
\begin{tikzpicture}[thick]
 \path (0, 0)  pic {fatjet=blue};
 \path (0, 0)  pic [rotate=-180] {fatjet=green};

 \begin{feynman}
  \vertex [dot] (origin) {};
  \vertex [right=1cm of origin, label={[red, xshift=-0.5cm, yshift=0.15cm]\(H\)}] (H);
  \vertex [above right=0.20cm and 2.3cm of H] (b1) {\(\bar{b}\)};
  \vertex [below right=0.20cm and 2.3cm of H] (b2) {\(b\)};
  \vertex [left=1cm of origin, label={[xshift=0.5cm, yshift=0.15cm]\(g\)}] (jet);
  \vertex [above left=0.20cm and 2.3cm of jet] (q1);
  \vertex [below left=0.20cm and 2.3cm of jet] (q2);
  \vertex [above left=0.40cm and 2.5cm of jet] (q3);
  \vertex [below left=0.40cm and 2.1cm of jet] (q4);
  \vertex [above left=0.05cm and 2.0cm of jet] (q5);
  \vertex [right=5cm of origin] (largeR_signal);
  \vertex [left=5cm of origin] (largeR_jet);
  \diagram* {
  (origin) -- [red, scalar, momentum'={}, style={pos=0.2}] (H) -- [anti fermion] (b1),
  (H) -- [fermion] (b2),
  (origin) -- [gluon, momentum={}] (jet),
  (jet) -- (q1),
  (jet) -- (q2),
  (jet) -- (q3),
  (jet) -- (q4),
  (jet) -- (q5),
  };
 \end{feynman}
\end{tikzpicture}
}
\caption{Cartoon showing columnated Higgs and ISR as a result of the large
boost due to their mutual recoil.} 
\label{fig:Hbb_ISR_FatJets}
\end{figure}

In pursuit of the rich physics potential discussed above an analysis of Boosted
Higgs signatures was undertaken and is discussed further in \Cref{part:HbbISR}.

