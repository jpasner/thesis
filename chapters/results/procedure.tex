\section{Measurement Procedure} \label{sec:results:procedure}

To measure the Standard Model signal of interest a model comprised of $V$ +
jets, $H \rightarrow b\bar{b}$ and $t\bar{t}$ templates along with a QCD
multijet model function is fit to the data. The $t\bar{t}$ template is
constructed from a MC sample with its normalization corrected with a $k$-factor
derived from a dedicated $t\bar{t}$ Control Region. This fit simultaneously
extracts the signal strengths of the $V$ + jets and $H \rightarrow b\bar{b}$
process ($\mu_{V}$ and $\mu_{H}$ respectively) which are parameterized with a
flat prior. The comparison of data to the maximum a posteriori probability
(post-fit) model is seen in \Cref{sec:results:money_plot} where the nuisance
parameters were constrained as shown in \Cref{sec:results:nuisance_parameters}.

\begin{figure}
\centering
\includegraphics[width=\linewidth]{figures/results/money_plot}
\caption{ 
The top panel shows the post-fit comparison of the signal candidate large-$R$
jet mass distribution for the combined SM Higgs boson, $V$ + jets, $t\bar{t}$
and QCD model to the observed data \cite{ATLAS-CONF-2018-052}.  The middle
panel gives the ratio of the post-fit model and the data with the QCD and
$t\bar{t}$ components subtracted, highlighting the large resonance from $V$ +
jets.  The bottom panel gives the ratio of the post-fit model and the data with
the QCD, $V$ + jets, and $t\bar{t}$ components subtracted, highlighting a
slight excess of events near $m_{J} = 125~\GeV$.}
\label{sec:results:money_plot}
\end{figure}

\begin{figure}
\centering
\includegraphics[width=0.7\linewidth]{figures/results/nuisance_parameters}
\caption{ 
Pulls of nuisance parameters in the final combined fit \cite{Krizka:2310645}.
}
\label{sec:results:nuisance_parameters}
\end{figure}
