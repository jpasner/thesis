\section{Trigger} \label{sec:atlas:trigger}

The 25ns spacing of proton bunches at the LHC results in a crossing rate of
proton bunches of 40MHz with roughly 33 \pp interaction events per crossing as
seen in \Cref{fig:pileup}.  Given that each event we store takes up 1.6 MB of
storage this would result in a data rate of 64 TB/s.  Given the limits of
technology and budget this rate of data is far beyond what we could read out,
much less store for analysis. The answer is the ATLAS trigger system which
decides in real time whether or not the current interaction contains something
we're interested in keeping, or whether to dump all sensor output and prepare
for the next bunch crossing.

ATLAS uses a two-level trigger system to reduce the event rate while making
sure to not skip any event of interest.  The Level-One (L1) trigger is hardware
based and uses coarse granularity inputs from the muon spectrometer and
calorimetery systems to make a decision within 2.5 $\mu s$ of the bunch
crossing for the event in question. This results in a event rate reduction from
40 MHz to 100 kHz.  The second level of the system, the High Level Trigger
(HLT), is software based and uses the full detector output to make its decision
within a 200ms window. This final triggering step reduces the event rate down
to a more manageable 1 kHz which is then written to disk at a rate of over a
Gigabyte per second.
