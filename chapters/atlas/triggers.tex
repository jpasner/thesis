\section{Trigger} \label{sec:atlas:trigger}

The $25~\ns$ spacing of proton bunches at the LHC results in a crossing rate of
proton bunches of $40~\MHz$ with an average of $31.9$ \pp interaction events per
crossing as seen in \Cref{fig:pileup}.  Given that each event takes up 1.6 MB
of storage this would result in a data rate of 64 TB/s.  Because of the limits
of technology and budget, this rate of data is far beyond what could be read
out, much less stored for analysis. The solution is the ATLAS trigger system
which decides in real time whether or not to record the current event, or dump
all sensor output and prepare for the next bunch crossing.

ATLAS uses a two-level trigger system to reduce the event rate while making
sure to not skip any event of interest.  The Level-One (L1) trigger is hardware
based and uses coarse granularity inputs from the muon spectrometer and
calorimetery systems to make a decision within $2.5~\mics$ of the bunch
crossing for the event in question. This results in an event rate reduction from
40 MHz to 100 kHz.  The second level of the system, the High Level Trigger
(HLT), is software based and uses the full detector output to make its decision
within a $200~\ms$ window. This final triggering step reduces the event rate down
to a more manageable 1 kHz which is then written to disk at a rate of over a
gigabyte per second.
