\section{Muon Spectrometer} \label{sec:atlas:muons}

The ATLAS Muon Spectrometer (MS) \cite{PERF-2007-01}, see figure
\ref{fig:muon_system}, accomplishes tracking of charged particles in the $|\eta|
< 2.7$ region for momentum reconstruction while also providing triggering on
charged particles in the $|\eta| < 2.4$ region.  The magnetic field necessary
for momentum reconstruction is provided by 3 air core torroid systems, one
barrel torrioid covering $|\eta| < 1.4$ and two endcap torroid systems which are
inserted into the inner radius of the the barrel torroid to cover the $1.6 <
|\eta| < 2.7$. The so called transition region $1.4 < |\eta| < 1.6$ between
these two magnet systems is covered by a combination of the barrel and endcap
torroid magnets.  Similar to the ID the resolution is inversely proportional to
the particle's incident momentum.  Any muon with pT lower than ~3GeV will never
make it to the MS and thus will not be detected.  

\begin{figure}[!htbp]
  \begin{center}
    \includegraphics[width=0.8\linewidth]{figures/atlas/muon_system}
    \caption{ \cite{PERF-2007-01} A cut-away diagram of the ATLAS muon system
and its many sub-detectors.}
    \label{fig:muon_system}
  \end{center}
\end{figure}

Precision tracking measurements for momentum reconstruction is accomplished
using the Monitored Drift Tube chambers (MDTs) for $|\eta| < 2.0$ and using
Cathode-Strip Chambers (CSCs) for $2.0 < |\eta| < 2.7$.  The MDT system consists of
1163 drift tube chambers arranged in three to eight layers for varying $\eta$.
The CSCs are designed to withstand the higher rate and retain good time
resolution using multiwire proportional chambers with orthogonal segmented
cathode planes.

The MS also gives nanosecond tracking information for triggering on muon tracks.
This is accomplished using Resistive Plate Chambers (RPC) in the barrel region
$|\eta| < 1.05$ and Thin Gap Chambers (TGC) in the end-cap $1.05 < |\eta| < 2.4$
region.  Both chamber systems deliver a triggerable signal with a spread of
$15-25$ ns, thus providing the ability to tag individual beam-crossings.
