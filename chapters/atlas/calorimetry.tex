\section{Calorimetry} \label{sec:atlas:calorimetry}

\begin{figure}[!htbp]
  \begin{center}
    \includegraphics[width=0.8\linewidth]{figures/atlas/calorimeter_cutaway}
    \caption{A cutaway diagram of ATLAS sampling calorimeters \cite{PERF-2007-01}.}
    \label{fig:calorimeter_cutaway}
  \end{center}
\end{figure}

Once the proton-proton collision remnants have passed through the ID and its
surrounding solenoid they enter into the ATLAS calorimeters depicted in
\Cref{fig:calorimeter_cutaway}.  Sampling calorimeter technologies were chosen
for their compact geometry and lower cost point.  These are constructed by
alternating layers of absorber, a dense material which reduces the incident
particles energy, and active material which produces a detectable signal when a
particle passes through.  This means that the detected signal is only a
fraction of the total energy of the particle and thus must be calibrated in a
study of the calorimeter \cite{Fabjan:692252}. The first system, the
Electromagnetic Calorimeter (EMC), is designed to measure the energy of
electrons and photons which primarily lose their energy via bremsstrahlung and
pair production electromagnetic interactions.  Outside of the EMC is the
hadronic calorimeter which is designed to measure the energy of jets of hadrons
through their electromagnetic and strong interactions. These detectors cover
the entire $|\eta| < 4.9$ range~\footnote{EMC performance is degraded in the
region $1.37 \leq |\eta| \leq 1.52$ due to the transition between the barrel
and the end-cap calorimeters} and provide complete containment of both
electromagnetic and hadronic showers with higher granularity in the EMC for
$|\eta| < 2.5$, the region matched to the ID, for precision measurements of
electrons and photons.

\subsection{Electromagnetic Calorimeter}

The innermost calorimeter, the Liquid Argon (LAr) Electromagnetic Calorimeter
(EMC) \cite{PERF-2007-01}, uses lead as the absorber and liquid argon as the
active material in an ``accordion geometry" as seen in \Cref{fig:accordion}.
This geometry was chosen for uniform coverage in $\hat{\phi}$ due to its lack
of un-instrumented cracks in the radial direction.  The barrel region covers
$|\eta| < 1.475$ and an end cap on each side covers $1.375 < |\eta| < 3.2$. The
barrel and end-cap calorimeters are each housed in their own cryostat.  The
barrel is composed of two half barrels with a $4~\mm$ gap at $z = 0$ and both
end caps are divided into an inner wheel covering $2.5 < |\eta| < 3.2$ and an
outer wheel covering $1.375 < |\eta| < 2.5$.

\begin{figure}[!htbp]
  \begin{center}
    \includegraphics[width=0.8\linewidth]{figures/atlas/accordion}
    \caption{Sketch of LAr EMC barrel module where the lead and liquid argon
layers are visible in an accordion like geometry. Looking from the foreground
to the back there are 3 different layers of readout cells visible
\cite{PERF-2007-01}.}
    \label{fig:accordion}
  \end{center}
\end{figure}

In the $|\eta| < 2.5$ region the EMC has 3 radial layers for precision physics
measurements.  Layer 1 consists of strip cells which are finely segmented with
$\Delta\eta = 0.0031$ and $\Delta\phi = 0.0245$ allowing for precision position
resolution which gives discrimination power between a single $\gamma$ deposit
and the $\pi^0$ characteristic $\gamma\gamma$ deposit. Layer 2, which collects
the largest fraction of energy from electromagnetic shower, is segmented with
$\Delta\eta = .025$ and $\Delta\phi = 0.0245$. Layer 3 collects the tail of the
electromagnetic shower using a coarser segmentation of $\Delta\eta = .05$ and
$\Delta\phi = 0.0245$.  Additionally, in the region $|\eta| < 1.8$ a thin
pre-sampler, which contains no lead absorber, was placed in front of Layer 1 to
allow for energy corrections due to losses upstream of the EMC.  Combined the
EMC is $>$ 22 radiation lengths ($X_0$) in the barrel and $>$ 24 $X_0$ in the
end-caps, where a radiation length is the average distance an electron travels
in a given material before losing $1/e$ of its original energy $E_0$ via
bremsstrahlung radiation.

\subsection{Hadronic Calorimeter}

Directly outside the EMC envelope is the hadronic calorimeter system
\cite{PERF-2007-01} which consists of three sampling calorimeter technologies:
the Tile Calorimeter, the LAr Hadronic End-cap Calorimeter (HEC) and the LAr
Forward Calorimeter (FCal).  Combined, these three subsystems give measurements
of hadronic jet energies in the $0 <|\eta| < 4.9$ range. The tile calorimeter
uses steel as the absorber layer and scintillating tiles as the active material
and covers the region $|\eta| < 1.7$ with a barrel section flanked by two barrel
extensions each divided azimuthally into 64 modules.  These scintillator tiles
are read out on two sides by wavelength-shifting fibers connected to
photomultiplier tubes as seen in \Cref{fig:tile_calorimeter}. At $\eta =
0$ the total tile calorimeter thickness is 9.7 nuclear interaction lengths
($\lambda$), where $\lambda$ is the average distance a hadron travels before
interacting inelastically with a nucleus.
\begin{figure}[!htbp]
  \begin{center}
    \includegraphics[width=0.8\linewidth]{figures/atlas/tile_calorimeter.pdf}
    \caption{Schematic of a tile calorimeter module
including a depiction of the connection between the scintillator tile to the
photomultiplier via a wavelength-shifting fiber \cite{PERF-2007-01}.}
    \label{fig:tile_calorimeter}
  \end{center}
\end{figure}

The high $\eta$ region receives a larger radiation dose that would damage
scintillator tiles. Thus, technologies that are more radiation hard were chosen
for the HEC \cite{CERN-LHCC-96-041}. The HEC is composed of two independent
wheels per end-cap located just past the EMC end-cap but sharing the same
cryostat. This system  uses copper as an absorber and liquid argon for the
active material and covers the $1.5 < |\eta| < 3.2$ range using 32 wedge-shaped
modules per wheel. Finally, the FCal shares the same cryostat as the EMC and
HEC end-caps and acts to extend the coverage of the combined calorimeter system
to include the $3.1 < |\eta| < 4.9$ range.  Each end-cap contains 3 modules,
the first an electromagnetic module (copper/liquid-argon) which is followed by
two hadronic modules which use (tungsten/liquid-argon).
