\chapter{The ATLAS Detector} \label{chap:atlas}

Given the immense energies availalbe at the LHC, and the veritable zoo of
paricles we are trying to detect, we require a general-purpose experiment in
order to fully exploit the full range of physics opportunities provided.  Two
international collaborations rose to this challenge, the CMS (Compact Muon
Solenoid) and ATLAS (A Torroidal LHC ApparatuS) experiments.  While both have
similar physics goals and each of them strengths and weaknesses, this
dissertation will focus on the ATLAS experiment and the intricacies of its three
main sub-detectors and two massive magnet systems depicted in
\cref{fig:atlas_cutaway}.

\begin{figure}[!htbp]
  \begin{center}
    \includegraphics[width=0.9\linewidth]{figures/atlas/atlas_cutaway.pdf}
    \caption{ \cite{PERF-2007-01} Here we see a cut-away side view of the ATLAS
detector with the major components labeled.  Note that within each of these
labeled components there may exist multiple different detector technologies.
For scale two people in red are shown standing between the disk muon chambers on the
left side of the figure. }
    \label{fig:atlas_cutaway}
  \end{center}
\end{figure}

Originally proposed in 1994 the ATLAS experiment was completed in 2008. On
July 4th, 2012 in a joint announcment the ATLAS and CMS experiments announced
the discovery of the long predicted Higgs Boson.  The collaboration now boasts
over 3000 physicists from 175 instituations spread accross 38 countries and
continues to probe the limits of the Standard Model in pursuit of answers to
some of Humanities deepest questions.

Located approximately 100 meters underground in a vast excavated chamber, the
ATLAS detector rests its 7000 metric tonnes on a bed of concrete reinforced
steel.  Out of it flows the signals of over 100 million electronic channels
through a zip tied mass of greater than 3000 kilometers of cabling.  At its very
center is one of the four interaction points of the LHC, specifically Point 1,
where the two counter circulating proton beams are skillfully shaped and then
collided by a series of magnets.  The energetic particles resultant from this
collision then fly out in all directions into the bulk of the ATLAS detector.

The first sub-system they meet is the Inner Detector (ID) and its many layers of
strip and pixel silcon detectors along with a transition radiation gaseous wire
detector, all bathed in the 2T mangnetic field of the surronding superconducting
solenoidal magnet.  This system exploits the ionization of charged particles to
track their curved trajectory through the magnetic field.  This curvature gives
us charge information, a momentum measurement, and precision 3D verticies
crucial to the identification of the secondary verticies of a b-hadron decay. 

Outside of the solenoid the particles are faced with first the Electromagnetic
and then the Hadronic sampling calorimeters. Here, layers of scintillator and
high radiation length materials are implemented to measure the energy of
electrons, photons, and hadrons. As the goal is to completely absorb the energy
of all outgoing particles the calorimeter has a nearly $4\pi$ solid angle
coverage.

Finally we have the muon system surrounding the calorimeter and equipped with
its own torroidal magnet system.  Here the charged muon bends in the magnetic
field while leaving a trail of ionization in the muon spectrometer before
exiting the detector completely.  Neutrinos are the only other standard model
particle that leave the detector, however they do so without detection.  A
depiction of the various particle interactions with the different detector
sub-systems can be seen in \cref{fig:detector_interactions}

In the following sections I will explain our choosen coordinate system and give 
a more detailed reveiw of these 3 detector sub-systems.

\begin{figure}[!htbp]
  \begin{center}
    \includegraphics[width=0.9\linewidth]{figures/atlas/detector_interactions}
    \caption{This slice of the ATLAS detector depicts how different particles
interact with each component of the detector it crosses.  A dashed line
indicates no interaction while a solid line indicates interaction. Electrons
(yellow/green) and charged hadrons (red) interact with the tracker and curve in
the solenoid's magnetic field.  Electrons and photons (yellow/green) are
absorbed by the Electromagnetic calorimeter.  All hadrons (red/yellow) are
absorbed by the Hadronic calorimeter. The muons (orange) curve in both the
solenoid and torroid magnetic fields before exiting the detector. Finally, the
neutrinos (white) pass through the entire detector without interacting.  }
    \label{fig:detector_interactions}
  \end{center}
\end{figure}

\section{ATLAS Coordinate System} \label{sec:atlas:coordinates}

Using the nominal interaction point as the origin, ATLAS uses a right-handed
coordinate system where the positive $x$-axis points towards the center of the
LHC ring, the positive $y$-axis points upwards, and the positive $z$-axis is
defined by the counter clockwise circulating beam direction as viewed from
above shown in \Cref{fig:atlas_geometry} \cite{PERF-2007-01}.  
 
\begin{figure}[!htbp]
  \begin{center}
    \includegraphics[width=0.5\linewidth]{figures/atlas/atlas_geometry}
    \caption{ \cite{Stark:2317296} A cartoon view of the the LHC from above
showing the SPS, LHC and the four main experiments of the LHC: ATLAS, CMS, LHCb,
and ALICE.  The standard cartesian coordinate system is shown with its origin at
the ATLAS interaction point, the positive $x$-axis towards the center of the
LHC, the positive $y$-axis pointing upwards, and the positive $z$-axis pointing
along the beamline towards the "A-side"}
    \label{fig:atlas_geometry}
  \end{center}
\end{figure}

Using these coordinates we can define the physical momentum of the objects
measured as $\vec{p} = (\pt,p_z)$ with \pt being the momentum of the object in
the transverse plane and $p_z$ the momentum along the beam axis. Given the
cylindrical symmetry of ATLAS it is desirable to define the polar angle
$\theta$ from the beam axis with the $r$-$\phi$ plane being perpendicular to that
axis. Since the particles we observe are relativistically boosted in the
$z$-axis it is desirable to use the Lorentz invariant quantity pseudorapidity
$(\eta)$ defined in terms of the polar angle by

\begin{equation}
 \eta = -\ln \tan \left( \frac{\theta}{2} \right).
\end{equation}

where $\eta = 0$ is in the $x$-$y$ plane and larger values of $|\eta|$ being
closer to the beam axis as can be seen in \Cref{fig:pseudorapidity}.

\begin{figure}[!htbp]
  \begin{center}
    \includegraphics[width=0.5\linewidth]{figures/atlas/pseudorapidity}
    \caption{Modified from \cite{Stark:2317296} this cartoon represents a
selection of pseudorapiditity $(\eta)$ values overlaid with some cartesian
coordinates (dashed black lines).  The red lines are drawn for $\eta = \pm
0.5,1.0,3.0$ }
    \label{fig:pseudorapidity}
  \end{center}
\end{figure}

In this analysis the angular separation between objects in the detector is
calculated and represented using the geometric quantity 

\begin{equation}
 \DeltaRdef
\end{equation}
 
\section{Tracking with the Inner Detector} \label{sec:atlas:tracking}

With its closest component, the insertable b-layer (IBL)
\cite{Potamianos:2209070}, only 3.3 cm from the beampipe The Inner Detector
(ID), shown in figure \ref{fig:inner_detector_diagram}
\cite{ATLAS-TDR-4,ATLAS-TDR-5} faces the incredible challenge of providing
precision momentum resolution and identification of both primary and secondary
vertex measurements of charged tracks all while recieving the highest fluence of any detector.

\begin{figure}[!htbp]
  \begin{center}
    \includegraphics[width=0.8\linewidth]{figures/atlas/inner_detector_diagram}
    \caption{ \cite{Potamianos:2209070} Diagram of inner detector}
    \label{fig:inner_detector_diagram}
  \end{center}
\end{figure}

It is designed to be very compact to fit inside the 2T solenoid and to give
excellent momentum resolution above the nominal \pT threshold of $0.5$GeV and
within the pseudorapidity range of $|\eta| < 2.5$ as shown in figure \ref{fig:inner_detector_schematic}

\begin{figure}[!htbp]
  \begin{center}
    \includegraphics[width=0.8\linewidth]{figures/atlas/inner_detector_schematic}
    \caption{ \cite{PIX-2018-001} Schematic of the Inner Detector including eta lines}
    \label{fig:inner_detector_schematic}
  \end{center}
\end{figure}

 
\section{Calorimetry} \label{sec:atlas:calorimetry}

\section{Muon Spectrometer} \label{sec:atlas:muons}

The ATLAS Muon Spectrometer (MS) \cite{PERF-2007-01}, see figure
\ref{fig:muon_system}, accomplishes tracking of charged particles in the $|\eta|
< 2.7$ region for momentum reconstruction while also providing triggering on
charged particles in the $|\eta| < 2.4$ region.  The magnetic field necessary
for momentum reconstruction is provided by 3 air core torroid systems, one
barrel torrioid covering $|\eta| < 1.4$ and two endcap torroid systems which are
inserted into the inner radius of the the barrel torroid to cover the $1.6 <
|\eta| < 2.7$. The so called transition region $1.4 < |\eta| < 1.6$ between
these two magnet systems is covered by a combination of the barrel and endcap
torroid magnets.  Similar to the ID the resolution is inversely proportional to
the particle's incident momentum.  Any muon with pT lower than ~3GeV will never
make it to the MS and thus will not be detected.  

\begin{figure}[!htbp]
  \begin{center}
    \includegraphics[width=0.8\linewidth]{figures/atlas/muon_system}
    \caption{ \cite{PERF-2007-01} A cut-away diagram of the ATLAS muon system
and its many sub-detectors.}
    \label{fig:muon_system}
  \end{center}
\end{figure}

Precision tracking measurements for momentum reconstruction is accomplished
using the Monitored Drift Tube chambers (MDTs) for $|\eta| < 2.0$ and using
Cathode-Strip Chambers (CSCs) for $2.0 < |\eta| < 2.7$.  The MDT system consists of
1163 drift tube chambers arranged in three to eight layers for varying $\eta$.
The CSCs are designed to withstand the higher rate and retain good time
resolution using multiwire proportional chambers with orthogonal segmented
cathode planes.

The MS also gives nanosecond tracking information for triggering on muon tracks.
This is accomplished using Resistive Plate Chambers (RPC) in the barrel region
$|\eta| < 1.05$ and Thin Gap Chambers (TGC) in the end-cap $1.05 < |\eta| < 2.4$
region.  Both chamber systems deliver a triggerable signal with a spread of
$15-25$ ns, thus providing the ability to tag individual beam-crossings.

