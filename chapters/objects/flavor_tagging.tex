\section{Flavor Tagged Jets} \label{sec:objects:flavor_tagging}

In general our reconstruction algorithms for jets are agnostic to the "flavor"
label - light ($l$), charm ($c$), or bottom ($b$) - of the hadrons inside of
the shower.  However, flavor tagging \cite{Aad:2015ydr} is a powerful tool for
discriminating the $b\bar{b}$ decay products of the Higgs from the large,
predominantly light-flavor, QCD background.  These $b$-quark initiated jets are
identified using the \texttt{MV2c10} \cite{ATL-PHYS-PUB-2015-022} Boosted
Decision Tree (BDT) \footnote{The name \texttt{MV2c10} means that this
multivariate algorithm had a training sample with roughly $\sim10\%$ $c$-jets and
$\sim90\%$ $l$-jets in order to give a good balance of $c$-jet and $l$-jet
rejection}, a machine learning algorithm that uses the weighted score from a
series of decision trees to give a discriminant for how similar any given jet
is to a $b$-jet.  The BDT uses inputs from the kinematics of the jet ($p_{T}$
and $|\eta|$), as well as the outputs of tracking algorithms, discussed below,
to look for signatures consistent with a $b$-hadron decay shown in
\Cref{sec:objects:b_decay}.  As we shall see tracking information is crucial to
flavor tagging and thus can only be applied within the tracking volume ($|\eta|
< 2.5$).

\begin{figure}[!htbp]
  \centering
  \includegraphics[width=0.8\linewidth]{figures/objects/b_decay}
  \caption{\cite{Chisholm:bjet} Cartoon of a $b$-jet decay containing a
$b$-hadron decay vertex (\textcolor{track_blue}{blue}~\protect\tikzdot{track_blue})
displaced from the primary $pp$ vertex (\textcolor{red}{red}~\protect\tikzdot{red}),
and a $c$-hadron decay vertex (\textcolor{orange}{orange}~\protect\tikzdot{orange})
further displaced and often close to the $b$-hadron flight axis. Here you can see the
secondary (\textcolor{track_blue}{blue}) and tertiary
(\textcolor{orange}{orange}) vertices have large impact parameters
(\textcolor{IPgreen}{green}) with respect to the primary $pp$ vertex.}
  \label{sec:objects:b_decay}
\end{figure}

The relatively long lifetime of $b$-hadrons ($\approx 1.5\text{ps}$) means that
they will travel a non-negligible distance ($\approx 5\text{mm}$) from the primary
interaction vertex before decaying.  This macroscopic flight distance is large
enough that this decay can be identified as a secondary vertex (SV) displaced
from the original primary vertex (PV). Furthermore, roughly 90\% of $b$-jets
will contain a $c$-jet which will create a tertiary vertex when it decays (TV).
The secondary vertex finding algorithms (SV1), and Kalman filter algorithms
(JetFitter) look for events matching this characteristic $b$-hadron decay
chain. 

Using tracking information the two-dimensional and three-dimensional impact
parameter algorithms, IP2D and IP3D, determine thetransverse and longitudinal
impact-parameters - $d_{\text{0}}$ and $z_{\text{0}}$ - respectively. Looking at
\Cref{sec:objects:impact_parameters} we see that the $b$-flavor jets tend to be
positive while $c$-jets and $l$-jets tend to be distributed more symmetrically
around 0.

\begin{figure}[!htbp]
  \centering
  \subcaptionbox{Transverse Impact Parameter $d_{0}$}{\includegraphics[width=0.48\linewidth]{figures/objects/IP3D_d0}} \hfill
  \subcaptionbox{Longitudinal Impact Parameter $z_{0}$}{\includegraphics[width=0.48\linewidth]{figures/objects/IP3D_z0}}

  \caption{\cite{ATL-PHYS-PUB-2017-013} Data-Monte Carlo comparisons of the
transverse ($d_{0}$) and longitudinal ($z_{0}$) impact parameter significance
values for IP3D selected charged tracks in the leading jet of a $Z\to\mu\mu +
\text{jets}$ dominated sample}
  \label{sec:objects:impact_parameters}
\end{figure}

In 2017 two new tools were included into the tagger
\cite{ATL-PHYS-PUB-2017-013}; a Recurrent Neural Network (RNN) impact parameter
tagger (RNNIP) and a Soft Muon Tagger (SMT).  The RNNIP
\cite{ATL-PHYS-PUB-2017-003} exploits the fact that $b$-jets tend to have many
tracks with highly significant impact parameters while $l$-jets do not
as seen in \Cref{sec:objects:RNNIP}.  The SMT \cite{Sciandra:2287545} searches
for muons coming from the semi-leptonic decays of $b$-hadrons and $c$-hadrons
to discriminate against $l$-jets which have much softer muons. The separation
power of the SMT is shown in \Cref{sec:objects:SMT}.

\begin{figure}[!htbp]
  \centering
  \subcaptionbox{$b$-jets}{\includegraphics[width=0.48\linewidth]{figures/objects/RNNIP_b_jets}} \hfill
  \subcaptionbox{$l$-jets}{\includegraphics[width=0.48\linewidth]{figures/objects/RNNIP_l_jets}}

  \caption{\cite{ATL-PHYS-PUB-2017-003} The distribution of the transverse impact parameter significance $S_{d0}$ for the leading $d_{0}$ significance track and subleading $d_{0}$ significance track for $b$-jets (left) and $l$-jets (right)}
  \label{sec:objects:RNNIP}
\end{figure}

\begin{figure}[!htbp]
  \centering
  \includegraphics[width=0.5\linewidth]{figures/objects/SMT}
  \caption{\cite{Sciandra:2287545} Normalized BDT response in simulated $t\bar{t}$ events of the SMT for reconstructed muons associated to $b$-jets (blue), $c$-jets, (green) and light-flavour jets (red)}
  \label{sec:objects:SMT}
\end{figure}

All of the above algorithm's outputs, along with jet kinematic information, are
used as input to the \texttt{MV2c10} algorithm as seen in
\Cref{sec:objects:BDT_flowchart}.  The output is a discriminant score which
indicates how $b$-jet-like or how un-$b$-jet-like the jet in question is,
compared to the training sample used, as shown in
\Cref{sec:objects:MV2c10_output}.  The performance is calibraed in data using
jets containing a muon, indicating the semileptonic decay of the $b$-hadron,
and a correction is derived \cite{Aaboud:2018xwy}.

\begin{figure}[!htbp]
  \centering
  \includegraphics[width=0.8\linewidth]{figures/objects/BDT_flowchart}
  \caption{\cite{Feickert:ML4Jets2018} Flowchart of inputs to the \texttt{MV2c10} $b$-tagging algorithm.}
  \label{sec:objects:BDT_flowchart}
\end{figure}

\begin{figure}[!htbp]
  \centering
  \subcaptionbox{\texttt{MV2c10} discriminant for $b$-jets as compared to $c$-jets and $l$-jets in simulated $t\bar{t}$ events.}{\includegraphics[width=0.48\linewidth]{figures/objects/BDT_output}} \hfill
  \subcaptionbox{$l$-jet and $c$-jet rejection factors as a function of b-jet tagging efficiency of the \texttt{MV2c10} BDT}{\includegraphics[width=0.48\linewidth]{figures/objects/b_jet_efficiency}}

  \caption{\cite{Aaboud:2018xwy}  Performance of the \texttt{MV2c10} BDT for
the 2016 optimization in simulated $t\bar{t}$ events.  The performance was
evaluated on $t\bar{t}$ events simulated using \textsc{Powheg} interfaced to
\textsc{Pythia6}}
  \label{sec:objects:MV2c10_output}
\end{figure}
