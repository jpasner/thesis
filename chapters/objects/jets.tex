\section{Jets} \label{sec:objects:jets}

The particles that carry QCD color charge do not exist by themselves in
isolation, but instead combine with other colored particles to form colorless
composite hadrons making it impossible to observe individual quarks or gluons
in ATLAS. This process of hadronization, discussed in \Cref{sec:theory:qcd},
creates a shower of charged and neutral particles that leave tracks in the
inner detector and energy deposits in the calorimeters.  Each $pp$ collision
event at the LHC tends to produce a lot of these hadronic showers which are
then clustered into objects known as ``jets."  These clustered jet objects thus
represent the reconstructed signature of a gluon or quark after it hadronizes.

Jets can be formed from tracks using the hits of charged particles in the ID,
from energy deposits left by both neutral and charged particles in the
calorimeters, or from the truth particles produced via MC.  Since there is
no unique procedure for clustering these signatures, several different
approaches have been developed. The most common clustering options are the
$k_{t}$, Cambridge-Aachen, and anti-$k_{t}$ algorithms \cite{Cacciari:2008gp}.
These algorithms work by iteratively applying the following rules to the
chosen collection of objects:

\begin{enumerate}
  \item Define the distance $d_{iB}$ between the object $i$ and the beamline $B$ in terms of the transverse momentum $\pT$ and $P$ which determines the effect of the energy in the clustering algorithm: \[ d_{iB} = p_{\text{T}i}^{2P} \]
  \item Compute the pairwise distance $d_{ij}$ between objects $i$ and $j$: \[ d_{ij} = \textrm{min}\left(p_{\text{T}i}^{2P}, p_{\text{T}j}^{2P}\right) \frac{\DeltaR_{ij}^{2}}{R^{2}} \] where $\DeltaR_{ij}^{2} = \left(\eta_{i} - \eta_{j}\right)^{2} + \left(\phi_{i} - \phi_{j}\right)^{2}$ is the geometric term dependent on rapidity $\eta$ and the azimuthal angle $\phi$.  $R$ is a parameter which controls the size of the jet.
  \item Choose the smallest distance $d_{\text{min}}$ out of the list of distances: \[ d_{\mathrm{min}} \in \left\{\left\{d_{ij}\right\},\left\{d_{iB}\right\}\right\} \] 
  \item If $d_{\text{min}}$ is the distance between an object, $i$, and the beamline: \[ d_{\mathrm{min}} \in \left\{d_{iB}\right\} \] label this object a jet and remove it from the list
  \item If $d_{\text{min}}$ is the distance between some objects $i$ and $j$: \[ d_{\mathrm{min}} \in \left\{d_{ij}\right\} \] merge objects $i$ and $j$ into a new object $k$, add $k$ to the object collection and remove objects $i$ and $j$.
  \item Repeat until all objects have been clustered into jets and the object collection is empty.
\end{enumerate}

In the above iterative process the choice of parameter $P$ corresponds to three
choices of jet clustering algorithm:

\begin{enumerate}

  \item[$P = 1$] This defines the $k_{t}$ algorithm \cite{Catani:1993hr}.  Here the algorithm prioritizes the clustering of soft objects first and then gradually moves on to harder objects.  This has the effect of creating irregular shaped jets that evolve from the ``outside-in" as seen in \Cref{sec:objects:kt}
  \item[$P = 0$] This defines the Cambridge-Aachen algorithm \cite{Dokshitzer:1997in}.  In this case the effect of the energy of the jet is ignored, and instead the jets are clustered based on their geometric distance only.  Again this results in irregular shaped jets as seen in \Cref{sec:objects:Cam_Aachen}
  \item[$P = -1$] This defines the anti-$k_{t}$ algorithm which was used in this analysis \cite{Cacciari:2008gp}.  This algorithm prioritizes the clustering of hard objects first and then clusters the surrounding softer objects.  This results in mostly cone-like jets which center on the hardest object(s) as seen in \Cref{sec:objects:anti_kt}.
\end{enumerate}

\begin{figure}[!htbp]
  \centering
  \subcaptionbox{$k_{t}$ \label{sec:objects:kt}}{\includegraphics[width=0.32\linewidth]{figures/objects/kt}}
  \subcaptionbox{Cambridge-Aachen \label{sec:objects:Cam_Aachen}}{\includegraphics[width=0.32\linewidth]{figures/objects/Cam_Aachen}}
  \subcaptionbox{anti-$k_{t}$ \label{sec:objects:anti_kt}}{\includegraphics[width=0.32\linewidth]{figures/objects/anti_kt}}
  \caption{Example clustering of a parton-level MC simulation event showing
clustered jet shapes for $R=1.0$ and (a) $P=1$, (b) $P=0$, and (c) $P=-1$
\cite{Cacciari:2008gp}.}
  \label{fig:clustering_algorithms}
\end{figure}

\subsection{Large-$R$ Jets} \label{sec:objects:fatjet}
This dissertation focuses on measuring the boosted signature of the Higgs boson
as it decays to $b\bar{b}$.  As discussed in \Cref{sec:higgs:boosted}, the
resulting decay products are highly collimated and thus can be reconstructed
using a single large radius (``large-$R$") jet.  These large-$R$ jets are
reconstructed from topological calorimeter clusters using the anti-$k_{t}$
algorithm with a radius parameter of $R = 1.0$ resulting in jets similar to
\Cref{sec:object:large_R_example} \cite{Aaboud:2018kfi, Aaboud:2017hdf}. After
clustering the large-$R$ jets are "trimmed" to improve mass resolution and
reduce dependence on pile-up. Trimming is done by first using the $k_{t}$
algorithm to recluster the constituents into subjets with $R=0.2$.  Then any
subjet with $\pT$ less than $5\%$ of the parent large-$R$ jet's energy is
removed \cite{Krohn:2009th}.

\begin{figure}[!htbp]
  \centering
  \includegraphics[width=0.7\linewidth]{figures/objects/large_R_example}
  \caption{\cite{ATLAS-CONF-2014-003} Simulation of calorimeter clusters for decay of a boosted $Z' \rightarrow t\bar{t}$ clustered in a large-$R$ jet.}
  \label{sec:object:large_R_example}
\end{figure}

\subsection{Variable Radius Track Jets} \label {sec:objects:vrjets}

After capturing the decay of the Higgs in a large-$R$ jet the next step is to
identify the two sub-jets that represent the $b$ and $\bar{b}$ children of the
Higgs. In this dissertation the Variable Radius (VR) jet was chosen
\cite{Krohn:2009zg,ATL-PHYS-PUB-2017-010} defined by a radius parameter
$R_{\text{eff}}$ which decreases as a function of the jet $\pT$:

\[ R_{\text{eff}}(\pT) = \frac{\rho}{\pT}. \]

The constant $\rho$ determines how quickly the effective size of the VR jet
decreases with respect to the transverse momentum contained inside the jet.  In
this definition the choice of $\rho$ should be proportional to the mass of the
resonance you are attempting to reconstruct and should correctly reproduce the
size of jets as long as $\rho \lesssim 2\pT$. In addition to $\rho$ the VR jet
algorithm requires two additional parameters, $R_{\text{min}}$ and
$R_{\text{max}}$, to impose lower and upper cut-offs on the jet size,
respectively.  These parameters are scanned to optimize the reconstruction of
track jets from $H \rightarrow b\bar{b}$. Using these bounds the
$R_{\text{eff}}$ becomes 

\[
 R_{\text{eff}} \left(\pT\right)= \text{max}\left[\text{min}\left(\frac{\rho}{\pT},R_{\text{max}}\right),R_{\text{min}}\right]\,.
\]

In reconstructing the Higgs boson with $m_{H} = 125~\GeV$, the variable radius
track jet parameters were chosen to be $\rho = 30~\GeV$, $R_{\text{min}}=0.02$,
and $R_{\text{max}}=0.4$, in order to maximize the truth subjet double
$b$-tagging efficiency \cite{ATL-PHYS-PUB-2017-010}, as seen in
\Cref{vr_parameters}.  In \Cref{sec:objects:average_deltaR} the VR Track Jets
properly describe the truth $\Delta R$ distribution for $b$-hadrons while the
$R=0.2$ fixed track jets deviate for higher Higgs jet $\pT$. This deviation is
caused by the subjets being so collimated that they begin to overlap.
Furthermore, in \Cref{sec:objects:leading_vr} and
\Cref{sec:objects:subleading_vr} the VR track jets do an equally good or better
job of identifying the subjets associated with $b$-hadrons when compared to the
$R=0.2$ fixed radius track jets.  This ability to give a flat efficiency across
the entire Higgs $\pT$ spectrum while accurately describing the topology makes
these VR track jets the perfect choice for this analysis.

\begin{figure}[!htbp]
  \centering
  \subcaptionbox{Study of $\rho$ VR track jet parameter with $R_{\text{min}} = 0.02$ and $R_{\text{max}} = 0.4$}{\includegraphics[width=0.48\linewidth]{figures/objects/rho}} \hfill
  \subcaptionbox{Study of $R_{\text{min}}$ VR track jet parameter with $\rho = 30~\GeV$ and $R_{\text{max}} = 0.4$}{\includegraphics[width=0.48\linewidth]{figures/objects/Rmin}} \hfill
  \subcaptionbox{Study of $R_{\text{max}}$ VR track jet parameter with $\rho = 30~\GeV$ and $R_{\text{min}} = 0.0.02$}{\includegraphics[width=0.48\linewidth]{figures/objects/Rmax}}
  \caption{Labeling efficiency of subjet double $b$-tagging using truth
information from Higgs decay as a function of Higgs jet $\pT$.  The efficiency
for $R=0.2$ fixed radius track jests is included for comparison.  Uncertainty
bars include statistical uncertainties only \cite{ATL-PHYS-PUB-2017-010}.}
  \label{vr_parameters}
\end{figure}

\begin{figure}[!htbp]
  \centering
  \includegraphics[width=0.98\linewidth]{figures/objects/average_deltaR}

  \caption{The average $\Delta R$ between either
the two leading truth $b$-hadrons or the two leading subjets associated to a
Higgs jet as a function of Higgs $\pT$ \cite{ATL-PHYS-PUB-2017-010}.}
  \label{sec:objects:average_deltaR}
\end{figure}

\begin{figure}[!htbp]
  \centering
  \subcaptionbox{$250~\GeV$ < Higgs $\pT$ < $400~\GeV$}{\includegraphics[width=0.48\linewidth]{figures/objects/low_leading_vr}} \hfill
  \subcaptionbox{$800~\GeV$ < Higgs $\pT$ < $1000~\GeV$}{\includegraphics[width=0.48\linewidth]{figures/objects/high_leading_vr}}
  \caption{Distributions of $\Delta R$ between a truth matched $b$-hadron and
the reconstructed leading subjet \cite{ATL-PHYS-PUB-2017-010}.  The
uncertainties given reflect only statistical uncertanties.  All algorithms are
normalized to an area corresponding to the fraction of signal jets which
contain a leading subjet.} 
  \label{sec:objects:leading_vr}
\end{figure}

\begin{figure}[!htbp]
  \centering
  \subcaptionbox{$250~\GeV$ < Higgs $\pT$ < $400~\GeV$}{\includegraphics[width=0.48\linewidth]{figures/objects/low_subleading_vr}} \hfill
  \subcaptionbox{$800~\GeV$ < Higgs $\pT$ < $1000~\GeV$}{\includegraphics[width=0.48\linewidth]{figures/objects/high_subleading_vr}}

  \caption{Distributions of $\Delta R$ between a truth matched $b$-hadron and
the reconstructed subleading subjet \cite{ATL-PHYS-PUB-2017-010}.  The
uncertainties given reflect only statistical uncertainties.  All algorithms are
normalized to an area corresponding to the fraction of signal jets which
contain a subleading subjet.} 
  \label{sec:objects:subleading_vr}
\end{figure}



