\chapter{Introduction} \label{sec:intro}
As children we are fascinated with the natural world that surrounds us.  How
was it made? What is it made of? Where did it come from?  When was it made?
These questions follow some of us into adulthood and lead to the study of the
fundamental interactions of the Universe, the pursuit of the building blocks of
reality.  For the past century, experiments of increasing size and complexity
have probed higher energies and smaller distances to answer these questions out
of the pure desire to know the unknown.  The results of these experiments have
been interpreted and used to build the most predictive and successful theory
across all of science - the Standard Model (SM) of particle physics. At the
current state-of-the-art facility, the Large Hadron Collider (LHC), an
international team of scientists and engineers continue this legacy of
discovery.  There, protons are smashed together at energies not seen since the
birth of the Universe some 13.8 billion years ago and the results are analyzed
to answer the outstanding questions of our time.

In 2012 the discovery of a Higgs-like boson \cite{Higgs:1964ia, Higgs:1964pj,
Higgs:1966ev, Englert:1964et, Guralnik:1964eu} at CERN by the ATLAS and CMS
\cite{Aad:2012tfa,Chatrchyan:2012xdj} collaborations answered one of these
outstanding questions, namely: What is the origin of mass for particles in the
SM?  This was a major triumph for both the theoretical and experimental
particle physics communities and resulted in the Nobel Prize in Physics being
awarded to Francois Englert and Peter W. Higgs for their contributions to the
Higgs Boson theory. However, the properties of this new particle are still
under verification and new physics could be lurking in the large uncertainties
of current observations of the SM Higgs boson.  In particular, the coupling
strength of the Higgs boson to bottom quarks ($H \rightarrow b\bar{b}$) was
only observed in 2018 and still contains a large uncertainty on the measurement
\cite{HIGG-2018-04, CMS:2018abb}. Furthermore, the $H \rightarrow b\bar{b}$
decay mode has yet to be observed for the vector boson fusion and gluon-gluon
fusion production mechanisms.  Finally, new physics could be accessible through
observation of highly boosted Higgs bosons produced via gluon-gluon fusion due
to its sensitivity to possible anomalous couplings in the top-quark loop
\cite{Grojean:2013nya, Schlaffer:2014osa}.  Thus, the goal of this analysis is
to directly measure the coupling of the Higgs boson to bottom quarks, with
focusing on the observation in the gluon-gluon fusion and vector boson fusion
production modes. 

\Cref{part:theory} of this dissertation describes the Standard Model of
particle physics including the Higgs mechanism and motivates the search for a
highly boosted $H \rightarrow b\bar{b}$. \Cref{part:experiment} describes both
the LHC and the ATLAS detector located at CERN. Lastly, \Cref{part:HbbISR}
presents the data analysis and the Boosted Higgs boson measurement results.
